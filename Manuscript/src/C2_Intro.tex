\chapter{Multirobot Systems and Automatic Design in Evolutionary Robotics}
\label{chapter:design}

% \epigraph{I need to find a clever epigraph.}{--- \textup{Arthur Bernard}}

\setcounter{secnumdepth}{2}
\setcounter{minitocdepth}{2}
\minitoc[n] % minitoc without title

In this Part of the manuscript, we study the automatic design of a distributed multirobot system. In particular, we are interested in the use of evolutionary robotics to design cooperative robots and the influence of genetic team composition~\parencite{Waibel2009} in the emergence of efficient coordination behaviours. Namely, we want to study the impact of genetically different robots (i.e. aclonal approaches)~\parencite{Quinn2001} on (1) the probability to evolve cooperation and (2) the efficiency of the cooperative solutions.

% as well as the evolution of specialisation by way of genotypic polymorphism.

Multirobot systems are many and widely diverse in the way they are designed. Multiple methods have been proposed for both the manual and automatic design of these systems. In this Chapter, we thus review the features (both negative and positive) of evolutionary robotics with regard to the design of distributed robotics in comparison to other techniques. First we give a quick overview of multirobot systems as well as their main advantages when compared to single robots. We emphasize on the design choices that come with building those systems. Additionally, we present a few applications of multirobot systems that are either seminal and/or noteworthy. Then we focus on the control of collective robotics systems. As there has been a strong interest in taking inspiration from single robots to apply the same solutions to multiple robots, we reveal the particular challenges brought up by these systems. We thus discuss the different manners in which multirobot systems have been manually designed with handcrafted behaviours. Then we address the automatic design of distributed robots thanks to machine learning and reinforcement learning in particular. Our goal is thus to shed some light on the advantages and limits of such approaches. We then move to evolutionary techniques and how they have been applied to the design of multirobot systems. We thus expose the main differences with classical machine learning in this context and quickly review the main results obtained in the design of collective behaviours. Finally, we present the open issue raised by choosing a clonal or aclonal team composition when designing cooperative robots.


\section{Multirobot Systems}
\label{sec:MRS}

  
    % A RECASER ?

    % CEBOT and ACTRESS are often cited among the earliest successful MRS. CEBOT~\parencite{Fukuda1988} (for CELlular roBOTic system) is a decentralized architecture inspired by cellular organization. The organization of the system is dynamic and robots, which are coupled to one another, can reconfigure their structure under environmental changes. This system is based on a hierarchical organization where "master cells" (which are also robots) can communicate with other master cells and allocate subtasks to all of the agents in the system. In comparison, in ACTRESS~\parencite{Asama1989} (ACtor-based Robot and Equipments Synthetic System) the system is composed of three robots and three workstations. One of these workstations is operated by an human, another is used as image processing and the last one manages the environment. Given this heterogeneous group of six agents, the goal is for the robots to perform a purely collective task (i.e. that could not be achieved by a single robot) like pushing an object. This system raises the issue of achieving efficient communication between multiple levels of organization.

    % While it may seem counter-productive to develop and control several robots where a single robot could very well be sufficient, using a team of robots has several advantages among which:

    % MRS can be constitued of between two to a thousand of autonomous robots~\parencite{Rubenstein2014} depending on the task at hand.

    % However it is important to note that because applications vary greatly, MRS are very different in design.


  \subsection{General Properties} 

    Multirobot systems (MRS), or sometimes multi-agent robotics~\parencite{Dudek1996}, essentially gained fame during the 1980s. The main motivation was to use cooperation between autonomous robots in order to cope with tasks that a classical single robot would not achieve. Multiple advantages may be obtained by using MRS~\parencite{Cao1997, Arkin1998a} :

    \begin{itemize}
      \item{The parallel execution of multiple robots allows the task to be achieved faster.}
      \item{Using multiple robots can ensure robustness and reliability through \emph{redundancy}.}
      \item{It can be both cheaper and simpler to produce several simple robots compared to a single complex one (especially if the robot may suffer damages).}
      \item{It may be necessary to distribute several robots at the same time to complete the task, in which case a single robot would simply not be sufficient.}
    \end{itemize}

    This implies that there are several crucial properties that are expected of MRS~\parencite{Parker1994}. First, MRS are supposed to be \emph{adaptable}. This means that each robot is expected to react to environmental change and, most importantly, to a change in others or induced by them. This also means that, in the lesser decentralized systems, the control system should change the global organization accordingly. Then, a MRS should be \emph{robust}. This implies that the system should not be critically impacted by failure (and in particular individual failures). This is easier said than done but this is also one of the main advantages of such an approach. Because we rely on multiple agents, it is possible to design the system so that a fault on one or several robots does not critically impact the whole system. Finally, it is often expected of MRS to be fully \emph{autonomous}. This means that the system as well as all the agents that compose it should be able to act without human intervention. In particular, the system should be able to face the unexpected without human control for some time.

    From these properties, it stems that there is a range of tasks that are especially appropriate for MRS. While these tasks often relate to real-world applications, they mainly represent general domains in order to design proof-of-concepts on a particular aspect of a MRS~\parencite{Parker2000}. We can briefly draw a list of the main tasks on which MRS are studied~\parencite{Cao1997, Parker2000, Farinelli2004} :

    \begin{description}
      \item[Foraging] {In this sort of tasks, the goal is to collect objects which are scattered in the environment. They then may or may not have to bring these back to a "home". Foraging refers to the real-world tasks of harvesting, toxic waste cleanup or search \& rescue. Most often, robots perform the task in a very independent manner, where the individual behaviour of a single entity does not really impact that of others. However, they may also rely on communication and \emph{stigmergy} in particular, i.e. indirect communication achieved by the previous modification of the local environment by an individual. Additionally, foraging tasks also have strong ties with biology and the behaviours of eusocial insects in particular. A main challenge in foraging is for the robots to efficienctly explore the environment, i.e. without repeating each other's actions.}

      \item[Collective transport] {The goal here is for several robots to collectively push an object (also called box pushing). This object is usually too big or too heavy for a single robot to move it alone and it thus requires the coordination of several individuals. A specific type of box pushing, cooperative manipulation, may require robots to carry the objects to a destination (rather than pushing). Box pushing may not necessarily require that robots be aware of the others for the task to be achieved~\parencite{Sen1994}.}

      \item[Collective motion] {A popular task is to design robots that are able to move in a coordinated manner. This may imply that we simply desire robots to move together towards a path, as in the case of flocking, or that we want them to adopt a particular formation for the duration of the motion. This sort of task can often be accomplished by agents with minimal capabilities in terms of sensors, effectors and communication. One of the central issues studied in collective motion is the design of simple and local (i.e. individual-level) control rules that allow for the collective emergence of the desired behaviour.}

      \item[Traffic control] {Another common task is that of traffic control. This is a problem of multirobot path planning, where several individuals often have their own personal goal. They must then coordinate with others in order to accomplish their goal without causing collisions or deadlocks. This is akin to a problem of resource conflict where robots have to share the environment with others.}

      \item[Monitoring] {Monitoring refers to the task of using multiple robots to observe and track a defined number of targets moving in the environment. Robots have thus to cooperate in order to ensure that all targets are monitored during the longest amount of time. In particular, there is a strong emphasis on coordination so that the agents can efficiently follow the targets and switch between them when necessary.}
    \end{description}

  \subsection{Architecture Choices}

    \begin{table*}[ht]
      \centerfloat
        \begin{tabular}{|l|c|c|c|c|c|c|}
          \hline
          \textbf{Control} & \multicolumn{3}{|c|}{Centralized} & \multicolumn{3}{|c|}{Decentralized} \\
          \hline
          \textbf{Team composition} & \multicolumn{3}{|c|}{Homogeneous} & \multicolumn{3}{|c|}{Heterogeneous} \\
          \hline
          \textbf{Communication} & \multicolumn{2}{|c|}{Environmental} & \multicolumn{2}{|c|}{Passive} & \multicolumn{2}{|c|}{Intentional} \\
          \hline
          \textbf{Group size} & \multicolumn{3}{|c|}{Small (\(\sim 10\))} & \multicolumn{3}{|c|}{Swarm (\(> 100)\))} \\
          \hline
        \end{tabular}
        \caption{\textbf{Architecture choices in multirobot systems.}}
      \label{table:architectureChoices}
    \end{table*}

    Because applications vary greatly, there is no canonical architecture for MRS~\parencite{Cao1997, Parker2008}. There are some more popular choices that we will highlight here but mainly one can use what best fits his/her needs. The main design choices w.r.t. architecture are the following (see also Table~\ref{table:architectureChoices} for a quick summary) :

    \begin{description}
      \item[Control] {The control of an MRS can be \emph{centralized} or \emph{decentralized}. In a centralized architecture, a single agent is responsible for controlling the system. Thus while this agent has full knowledge of the whole system, it represents a critical point for failures. Therefore, this type of organization is rare in MRS and most use a decentralized approach~\parencite{Parker2008}. However, the work of D'Andrea~\parencite{DAndrea2012} on the development of the Kiva systems, where a large group of robots (hundreds) move in a wharehouse to bring products to the workers, is of note in this category. In particular, a central control was responsible for the coordination of all the robots. 

      In comparison, decentralized architectures can be of two types: \emph{hierarchical} or \emph{distributed}~\parencite{Cao1997}. In a hierarchical architecture, the system is locally centralized and some agents are in charge of a group of other agents to organize the task at hand. For instance, in one of the very first successful MRS, CEBOT~\parencite{Fukuda1988}, particular robots (called "master cells") could communicate with other master cells and allocate subtasks to all of the agents in the system. On the contrary, in a distributed system, all agents are equal w.r.t. control which, while robust, implies that it is harder to achieve coherence between every agent.}

      \item[Team composition] {It is possible to use \emph{homogeneous} or \emph{heterogeneous} groups of robots. In an homogeneous team, individuals are all identical in terms of both software (control) or hardware (morphology and sensors). In comparison, heterogeneous robots vary between one another on any or both of these aspects. In consequence, homogeneous teams are more resilient to failures as every agent has the same capabilities, thus decreasing the impact on the system of losing a given individual. It is also easier to allocate tasks between robots because every agent can perform equally. However heterogeneous groups allow to benefit from differences between individuals to achieve more diverse behaviour, in particular when coordination is required.}

      \item[Communication] {We can mainly divide the type of communication implemented in MRS into three categories: \emph{environmental}, \emph{passive} and \emph{intentional}~\parencite{Cao1997, Parker2008}. Environmental communication refers to the indirect communication we briefly mentioned previously: stigmergy. Stigmergy means that the agents will sense modifications in the environment done by a previous agent and use this information to modify their behaviour. It is as if one agent indirectly communicated to the other through the environment. This type of communication is thus limited by the capabilities of agents to perceive complex information from the environment. Passive communication is another type of indirect communication where the agents rely on their sensors to observe the actions of others in the group. Thanks to this sensory feedback, robots can interact with each other without needing direct communication. However, it shares the same limitations as with stigmergy. Finally, intentional communication refers to direct communication between the robots. This thus allows to exchange complete information between teammates.}

      \item[Group size] {On this point, MRS are really diverse and the number of robots involved in a collective task can scale from two to a thousand~\parencite{Rubenstein2014}. A smaller group size usually means that it is possible to design more morphologically complicated robots where each individual may have elaborate capabilities. On the contrary, when we are interested in bigger teams, individual capacities tend to decrease in favour of collective complexity. However, an open challenge is to scale up algorithms that were designed for a small group of robots to a larger team. Large groups of robots are often referred to as \emph{swarms}~\parencite{Beni2005}. In this case, robots in a swarm often possess very basic sensory capabilities and may not achieve much on their own. The emphasis is put on the \emph{emergence} of collective functionnalities from individual interactions~\parencite{Kube1993, Parker2008}. This means that we expect to see the appearance of global collective complexity from the local interactions between agents of the swarm, a process also known as \emph{self-organization}. More precisely, swarm robotics are based on the principle of superadditivity~\parencite{Parker2008}, where the whole result (collective behaviour) is better than a simple sum of all its parts (the agents' behaviours).}
    \end{description}

    In this manuscript we study a particular instance of MRS. Namely, we focus on the design challenges of fully distributed MRS (i.e. decentralized) where a small group of robots are morphologically homogeneous. In this context, we are interested in the critical differences that come with using a team of homogeneous or heterogeneous robots w.r.t. control. Lastly, as communication is not the focus of our study, we do not explicitly give the robots particular communication capabilities. Therefore, if any communication takes place between individuals then it can be considered to be passive. The rest of this Chapter is to be understood in the context of these specific architecture choices.

    % Ajouter que nous on s'intéresse qu'aux cas où ya de la coop/coord ?

  % \subsection{Open Challenges}

      % Je pourrai mettre un challenge qui est l'origine de la coopération et donc recaser ce que j'avais déjà écrit


  %   % One critical design choice we need to address given this thesis is team composition, which may be \emph{homogeneous} or \emph{heterogeneous}. In an homogeneous team, individuals are all identical in terms of both software (control) and hardware (morphology and sensors). In comparison, heterogeneous robots vary grealt between one another. Most works in MRS have often focused on homogeneous teams as it is more practical in terms of task allocation: because every agent is the same, they all can achieve the same tasks. In consequence it is also more resilient to failures. In comparison, it is more difficult for heterogeneous teams to achieve coordination~\parencite{Parker1994}. However, they can benefit from the differences between individuals to display more diverse coordination strategies. Designing heterogeneous systems gives rise to critical issues in the field of MRS~\parencite{Parker2008}:

  %   \begin{itemize}
  %     \item{How to achieve efficient communications between several different robots ?~\parencite{Jung2000}}
  %     \item{How to efficiently allocate tasks between agents with differing capabilities ?~\parencite{Parker2003}}
  %   \end{itemize}

  %   % À réfléchir: un bla bla sur les différentes façons de faire du task allocation et les différentes formes de communication ? Je pense pas que ça soit nécessaire mais si on a de la place... pourquoi pas ?

  % \subsection{The Origin of Cooperation} 

  %   One of the main problems at the heart of designing a MRS is: how to achieve cooperation ? In their popular (though now ancient) review of the field, Cao and colleagues~\parencite{Cao1997} deemed this as one of the prime research axis in MRS. McFarland~\parencite{McFarland1996} argued that the design of cooperative robots could fall into two broad categories that he considered to be the same for group behaviours in nature: \emph{cooperative behaviour} and \emph{eusocial behaviour}. While this classification is more than doubtful for natural cooperative behaviours, its biological validity is not of real importance here. The more important point here is that since nearly the beginning of MRS, there has been an interest in taking inspiration from natural social behaviours for achieving coopration. Although without direct biological analogy, Parker~\parencite{Parker2008} classified the design of multirobot cooperation in two similar categories: \emph{intentionally cooperative} systems and \emph{collective swarm} systems. This categorization entails different manners in which to ensure cooperation.

  %   % but [McFarland] was mainly basing his argument on the work of Tinbergen~\parencite{Tinbergen1953} -> Really ?

  %   The intentionally cooperative MRS mostly comprised systems where agents have high knowledge about the other individuals' presence. They are capable of acting in accordance to others' actions and capabilities and may use communication to coordinate. In these MRS that McFarland simply classified as "cooperative behaviours", he defined an agent to be selfish. This means that cooperation comes from the maximization of the agent own utility. This paradigm is often caracterized by a more direct approach to cooperation. In particular, there is careful design on the manner with which robots can coordinate. Moreover, this approach is often well suited for MRS dealing with groups of heterogeneous robots, where the origin of cooperation represents a challenge by itself. From this it stems that this type of MRS sometimes takes inspiration from \emph{distributed artificial intelligence} (DAI). This field is mainly concerned with the design of distributed systems of intelligent agents~\parencite{Cao1997, Panait2005} and is generally considered to be divided in two major areas of studies: distributed problem solving (DPS) and multiagent systems (MAS). To summarize quickly, DPS is mostly concerned with solving problems with several agents. As such, some of its problematics are common to MRS (e.g. task allocation). However, because in DPS agents are considered to always be cooperative and are usually disembodied, few works can really contribute to MRS. In comparison, in MAS agents are often rational and as such are not de facto cooperators. Thus there is a strong emphasis on the collective interactions between agents. This is a reason why there has been consistent interest in game-theoretic approaches with MAS~\parencite{Rosenschein1985}. Therefore, MAS carry some theoretical ground for achieving cooperation in MRS. However, some have argued that MAS are not rooted enough in the physical world for them to make a powerful contribution to MRS~\parencite{Cao1997, Farinelli2004}. In particular, perfect sensory information is often assumed in MAS which may hinder direct transfer to robotics. For this reason, research in DAI tend to consider the mechanisms of coordination behaviours as a black box~\parencite{Parker1994}.

  %   % We believe selfishness is not necessarily associated with intionnally cooperation ?


  %   On the other end of the spectrum lies collective swarm~\parencite{Beni2005}, also called collective robotics~\parencite{Kube1993, Parker2008}. Although it could be argued that any MRS is a particular instance of collective robotics, we will make here the distinction as to not generate confusion w.r.t. the literature. In this approach, the MRS is often constituted of a high number of robots (at least several dozens) which are all homogeneous and distributed agents. Also called "reactive collective robotics", collective swarm takes inspiration from the behaviours and organization of eusocial insects~\parencite{Wilson1998, Werfel2014}~\footnote{A more thorough presentation of eusociality and the altruistic behaviours of eusocial animals is given in Chapter~\ref{chapter:model}. Consequently, there will not be an extensive discussion on this subject here.}. More precisely, the study of eusocial insects led to the creation of the field of Swarm Intelligence~\parencite{Bonabeau1999, Zoghby2013}. This field consists of heuristics designed to solve algorithmic problems by taking inspiration from the natural collective behaviours. Some of the most famous algorithms that stemed from Swarm Intelligence are ant colony optimization (ACO)~\parencite{Dorigo2004a}, for which the most classical problem is to find the best travel route in a traveling salesman problem, and particle swarm optimization (PSO)~\parencite{Kennedy1995}, where candidate solutions to an optimization problem are modeled as particles moving through the search space. The main goal of swarm robotics is to design a large colony of decentralized and self-organized robots capable of high flexibility and robustness.

  %   Agents in a swarm are as simple as possible and constituted of very basic sensory capabilities. In particular, direct communication between robots is often inexistent. Instead, they rely on \emph{stigmergy}, where indirect communication is achieved by looking at other individuals' modification of the environment (e.g. the use of pheromones by ants in the natural world). Additionally, robots in a swarm are often largely unaware of the actions and internal states of others, basing their knowledge on proximity information. This implies that robots are not capable of achieving much on their own. Their role is really to be a part of the collective. In particular, the main concept of swarm robotics is that of \emph{emergence}\footnote{It is interesting to note that swarm robotics and individual-based modeling, which we presented in Chapter~\ref{chapter:model}, share a lot of similar key concepts. In particular, the concept of emergent collective behaviours between agents is one that is central to IBM.}. This means that we expect to see the appearance of global collective complexity from the local interactions between agents of the swarm, something which is also known as \emph{self-organization}. More precisely, swarm robotics are based on the principle of superadditivity~\parencite{Parker2008}, where the whole result (collective behaviour) is better than a simple sum of all its parts (the agents' behaviours). In consequence, the design of a swarm is focused on creating simple local behavioural rules that should allow the whole system to act in a collective way. It is as if cooperation is a side effect resulting from individual behaviours. This is obviously complex to design and time must not be critical. However, it also means that agents are cheaper to produce, deploy and control. Also, thanks to self-organization, the tasks covered by swarm robotics often require little to no \emph{a priori} assumptions. At the inception of collective robotics, this conception was really different from the "classical" design paradigm in robotics and especially in AI which emphasized on high reasoning and higher-levels of cognition~\parencite{Bonabeau1999}. While examples of swarm robotics systems are numerous we can quickly name a couple. One of the first examples of successful swarm robotics on real robots was the \emph{Nerd Herd} by Matarić~\parencite{Mataric1995}. With a group of $20$ identical robots with very simple individual capabitilies (mainly detection of obstacles and other robots) and a set of pre-programmed behaviours, she created a system capable of collective behaviours of flocking, surrounding, herding and foraging. Another interesting example is that of Swarm-bot~\parencite{Mondada2004, Dorigo2004, Mondada2005}. The goal was to engineer a swarm of simple identical robots capable of using self-assembly to navigate accross rough terrain and achieve different collective tasks.

    % -> This is not always that simple. Genre en swarm on peut faire plein de trucs globalement et des fois ya du knowledge (je crois). Et le swarm peut être classé en MAS (voir Panait2005)
    % -> Globalement frontière blurry encore une fois. On va assez vite parler de swarms pour des trucs qui répondent pas aux principes de base des swarms


\section{Designing the Control of Collective Robots}

  So far we have presented a general overview of the properties and choices that come with designing a MRS. Namely, we discussed the \emph{what} and \emph{why} of multirobot systems. In this Section, we focus on the \emph{how}. More precisely, we are interested in the design techniques involved in creating a distributed multirobot system. Designing the control of a robot mainly depends on its situatedness, i.e. the complexity and uncertainty of the environment in which it operates~\parencite{Mataric2008}. Most of the early design techniques used in MRS have been inspired by classical single robots techniques.

  \subsection{The Deliberative Approach}

    This approach is also referred to as the Sense-Model-Plan-Act architecture~\parencite{Albus1991, Iocchi2001, Mataric2008}. This has been the classical approach in robotics~\parencite{Nilsson1984} and AI and is concerned with representing high reasoning capacities. As such, it has also historically been one of the first manners in which to approach MRS design. The basic principle is that all sensory information is computed under the internal knowledge of the robot in order to plan and determine the next action. This means that these architectures are based on an internal representation of the world. The model is often constituted of a set of symbols which are computed by a logical system. However, planning is a classical problem in AI and is known to be time costly. Therefore, while this architecture would be the most efficient in a perfect world, the process of building a world representation and planning is computationally expensive and lacks critical real-time reactivity.

    In the case of multirobot systems, we may define an additional global level of control as \emph{social deliberative}~\parencite{Iocchi2001}. In a social deliberative MRS, a global strategy will be planned so that the organization of the whole system (e.g. task allocation) can handle environmental changes. This type of MRS may have a global representation of the world shared between the agents but it is not necessary. Do note that, as we will see, the global control design of the MRS often differs from that of the individuals (e.g. a social deliberative system may be composed of behaviour-based robots).

    Mostly, when a deliberative approach is adopted, it is used as a global level control only. Werger and Matarić~\parencite{Werger2000} designed a MRS tasked with multi-target observation (i.e. monitoring) with a social deliberative distributed MRS. While robotic agents were individually behaviour based (see next Section), group-level deliberation was achieved thanks to an architecture they called the "Broadcast of Local Eligibility". More precisely, each robot could evaluate its eligibility to accomplish a given task and then broadcast this value. The robot with the highest value could then claim the task. They were thus able to solve the issue of task assignment in their system. In the context of the RoboCup~\parencite{Kitano1997}, a soccer competition between teams of robots, Candea and colleagues~\parencite{Candea2001} developed a distributed and heterogeneous soccer team in a deliberative system. In their proposed architecture, each robot was capable of playing any role and they could switch during a match. In particular, a voting system coupled with the possibility to evaluate and communicate the utility of an agent for a given role was used to efficiently assign these roles to the individuals. They could thus achieve strong coordination between distributed and heterogeneous robotic agents.

    %In particular, keeping an accurate internal representation up-to-date proves to be difficult in dynamic and noisy environment.

  \subsection{Manual Design}

    Because deliberation alone is hard to achieve in a distributed multirobot context, there has been a strong interest in using ad-hoc methods for the programming of robots behaviours.

    \subsubsection{Reactive approaches}

      % Gaffe à ce que je me répète pas par rapport à quand je parle d'emergence avant

      In light of the complexity of deliberative architectures was born the opposite stance: the \emph{reactive approach}~\parencite{Brooks1986}. In comparison with deliberation, this architecture is not based on reasoning nor planning. Rather, there is a direct connection from sensors to effectors, inspired by the biological concept of stimulus-response. This architecture is usually constituted of a programmed set of rules which, given the sensory inputs, return the desired output actions. This implies that reactive systems can achieve very fast computation and thus are convenient when quick reaction is necessary. However, as robots do not keep any representation of the world and most often do not store any information, they are basically myopic. This can be useful when \emph{a priori} knowledge of the environment is sufficient but does not fare well with uncertainty and novelty. Additionally, there is a strong emphasis on the concept of emergence we previously mentioned. Because the control of agents under a reactive approach does not allow for complex individual capacities, we expect functionalities to come from the emergent collective behaviour. This means that designing a reactive distributed system is a bottom-up approach which often implies a back and forth between handcrafting individual behaviours and observing the collective result.

      A popular way to design reactive behaviours is also to take inspiration from nature. For instance, the well-known work of Reynolds on the "boids"~\parencite{Reynolds1987} was inspired by the collective motion of flocks of birds. He developed a simulation of the collective behaviours of a swarm of simple agents with no cognitive capabilities under three ordered basic rules: collision avoidance, velocity matching and flock centering. Namely, agents should avoid collisions with neighbours, match their direction and speed (i.e. velocity) with that of others in the flock and stay close to other individuals. He was able to thus design a group of agents behaving in a similar way as a flock. Hauert et al.~\parencite{Hauert2011} implemented Reynolds' boids on 10 real flying robots. In particular, they were interested in the influence of communication range and turn rate on the emergence of flock-like behaviours. They showed in simulation that communication range needed to be high enough so that coherent flocks could be maintained. Rubenstein and colleagues~\parencite{Rubenstein2014} used a swarm of a thousand individuals capable of self-assembly. The robots, called kilobots, were embedded with an infrared transmitter and receiver to communicate with neighbours and measure their proximity. The system was given a 2D image by a user defining the shape the robots must replicate. One after the other, the agents then used very simple local rules by following the edge of the group and tracking the distance from their origin to collectively organize into the desired shape.

      % Viscek ?

      % Then, \emph{hybrid approaches} have been proposed, whose goal was to unite the best of both worlds. Namely, they attempt to combine the speed of response of reactive approaches and the optimal planning of delibarate approaches. Such architecture is basically constituted of three layers~\parencite{Mataric2008}. One layer is responsible for the reaction and execution of the robot, another layer is concerned with delibaration and planning and the third and final layer acts as an intermediate between the two others. In consequence, one of the main complexity in this approach is to design this last layer. It should indeed coordinate between the immediate needs dealt by the first layer and the more long-term decision of the second.

    \subsubsection{Behaviour-based approaches}

      Behaviour-based approaches~\parencite{Arkin1998a} were proposed after the appearance of reactive approaches, with the desire to improve on the latter's capacity to react to dynamic environments. This was mostly introduced and popularized by Rodney Brooks subsumption architecture~\parencite{Brooks1986}. In behaviour-based architectures, the robot control is constituted of several basic behaviours, which are organised in separate modules. In a similar way as the reactive approach, these behaviours are directly connected to the sensors and will activate according to a certain set of rules. However, in comparison to a purely reactive approach, these behavioural modules can keep a state as well as a representation of the world, allowing for higher reasoning and planning. Additionally, they are connected to each other in a hierarchical fashion.   Those modules are designed to interact with one another in order to collectively achieve the task at hand. Complexity is thus expected to emerge from the interactions between low-level behaviours. In consequence, behaviour-based architectures are efficient when the environment is dynamic but pure reactivity alone is not sufficient. In a similar way as reactive control, these architectures are usually designed in a bottom-up approach where behaviours are coded incrementally as building blocks in an increasing complexity. 

      Behaviour-based approaches are among the most used for handcrafted robot control in MRS~\parencite{Arkin1998a, Mataric2008, Parker2008}. For instance, Parker proposed and developed the ALLIANCE architecture which she successfully implemented on real robots~\parencite{Parker1994}. The issue was to design a fault-resistant system of heterogeneous robots which could achieve coordination. Based on a subsumption architecture, the system was composed of low-level behaviours combined together in a set to accomplish a particular task. Then, given environmental information (e.g. which task is being taken care of, what needs to be done), a motivation was computed to select the appropriate behaviour set. Matarić proposed the \emph{Nerd Herd}~\parencite{Mataric1995}, a group of $20$ identical robots capable only of detecting obstacles and other robots. Each robot was constituted of the same set of pre-programmed behaviours as obstacle avoidance, homing, aggregation, dispersion, following and safe wandering. The system could combine these behaviours in order to achieve higher functionnalities. For instance, collective foraging was obtained by applying a temporal combination operator in order to switch between avoidance, dispersion, following, homing and wandering. 

      % Proposer un exemple de plus ?

      % Brambilla and colleagues~\parencite{Brambilla2012} proposed a top-down property-driven method for swarm design. This method rests upon defining a set of properties that must hold true when  

      % In comparison, Soysal and colleagues~\parencite{Soysal2007} used probalistic finite state machines to switch between the adequate behaviours. Robots were composed of four basic behaviours: obstacle avoidance, approach, repel and wait.


  % Citer des architectures délibératives classiques ? (genre STRIPS ou NOAH)

  %One main challenge of this approach is to design the action selection, which is the process thanks to which the system will choose which behaviour choose from several~\parencite{Pirjanian2000}. Two popular ways to solve this problem are to either base the selection on a prefixed hierarchy between modules or to rely on a voting mechanism.

  % This approach shares similar features with the reactive approach.

  % In the case of multirobot systems, there is an additional global level of control that needs to be taken into account. This global control is characterized as either \emph{reactive} or \emph{social deliberative}~\parencite{Iocchi2001, Carpin2001}. Those concepts are similar to those of single robot control. In a reactive MRS, each individual deals with environmental changes without the influence of a higher control. In consequence, there is no model of the environment in the system and each agent is expected to individually adapt. In comparison, in a social deliberative MRS, a global strategy will be planned so that the organization of the whole system (e.g. task allocation) can handle environmental changes. This type of MRS may have a global representation of the world shared between the agents but it is not necessary. It is important to note that the global architecture of the MRS may differ from that of the individuals (e.g. a social deliberative system may be composed of behaviour-based robots).

  % The adaptability and simplicity of behaviour-based control is a critical advantage for creating cooperative tasks and achieving coordination~\parencite{Mataric1995, Iocchi2001}.  Then, Candea and colleagues developed the ART architecture~\parencite{Candea2001} in order to design a team of robots to participate in the RoboCup competition~\parencite{Kitano1997}. The RoboCup is an annual competition where teams of robots must compete in a soccer game, which thus presents several critical challenges for MRS (e.g. collaboration, robot control, reasoning) in an adversarial context. The ART architecture in particular was composed of a team of distributed heterogeneous robots which differed both in hardware and software. Robots could vote on team formation. They would then assign roles by communicating and evaluating the utility of a given role for a given robot. Finally more recently, in an article published in Science, Werfel et al. designed independent robots capable of building structures whose blueprint is given by a user~\parencite{Werfel2014}. They used a team of homogeneous robots, taking inspiration from the mount-building capabilities of termites. These robots could communicate through stigermy (i.e. indirect communication) and followed building rules automatically compiled by the system. The agents were fully reactive so that they could adapt to a change in the current structure (either from robot or human action). Given this last example, it is also interesting to briefly discuss where collective robotics (i.e. swarms) stand in terms of architecture. As previously said, the similarities between reactive and behaviour-based architectures with swarms are numerours. In particular, they both focus on low-level, local and individual rules so that a global collective can emerge. For these reasons, all collective robotics architecture, when they are not automatically designed (which we will talk about next), are behaviour-based~\parencite{Brambilla2012, Zoghby2013}.

  \subsection{Learning for Automatic Design}

    In comparison to the approaches presented in the previous Section, there has been also a strong interest in using automatic design for MRS. In particular, handcrafted behaviours do not fair well in the face of uncertainty and varying environmental conditions and may require a tedious back and forth before the adequate collective behaviour is obtained. In consequence, there is a large body of work on using learning for the design of MRS. For instance, in an article published in Science, Werfel et al.~\parencite{Werfel2014} focused on real robots capable of building structures. They used a team of homogeneous robots, taking inspiration from the mount-building capabilites of termites. The goal was to achieve the construction of a user-specified structure with $3$ robots and small identical bricks. While the robots were fully reactive and relied only on stigmergy, the rules they followed were automatically generated. This was done by an offline compilation process which, given the building structure, generated a representation of movement guidelines for robots, akin to traffic laws. They showed that their group of robots was able to collectively construct the structures as well as dynamically react to changes in the built structure.

    However, most learning techniques in MRS rely on machine learning. Machine learning has always been a critical challenge in artificial intelligence. Thus it has naturally been applied to robotics~\parencite{Hertzberg2008}. Classical machine learning can be divided into three different categories: supervised, unsupervised and reward-based. While the goal in machine learning is generally to optimize performance (e.g. for classifiers), the emphasis in mobile robotics is that the robot may adapt quickly. As such, most of the literature on learning in robotics has been focused on reward-based techniques~\parencite{Mataric2008}, most commonly referred to as \emph{reinforcement learning} (RL)~\parencite{Sutton1998}. RL rests upon the mathematical framework of markov decision processes (MDP)~\parencite{Bellman1957}. In RL a robot learns an optimal policy (i.e. a sequence of actions depending on the states the robot is in) thanks to a value function. Learning is thus achieved through rewards and punishments attributed to the robot according to its actions. The general goal in RL is to estimate the value function. This value function corresponds to the expected value of a state given a certain policy.

    % In particular, RL is based on the model of markov decision processes (MDP)~\parencite{Bellman1957}. To summarize quickly, a MDP is constituted of a set of states $S$ as well as a set possible actions $A$ given each state. Additionally, a transition function $P_{a}(s,s')$ which, given a certain state $s$ and action $a$ at time $t$ represents the probability to be in state $s'$ at $t+1$. Finally, $R_{a}(s,s')$ represents the reward obtained by transitioning from $s$ to $s'$ thanks to action $a$. The goal of a MDP is to find the optimal policy $\pi$, where $\pi(s)$ indicates the action to choose in state $s$, which maximizes:

    % \[
    %   \sum_{t=0}^{\inf} \gamma^{t}R_{a_{t}}(s_t,s_{t+1})
    % \]

    % where $\gamma$ is a discount factor. 

    % In RL, the goal is generally to estimate the value function $V^{\pi}(s)$, which corresponds to the expected value of the state $s$ given that we follow the policy $\pi$ afterwards. This means that $V$ is obtained as follows:

    % \[
    %   V^{\pi}(s)=E_{\pi}\left\{\sum_{k=0}^{\inf} \gamma^k r_{t+k+1} | s_t = s \right\}
    % \]

    % where $E_{\pi}$ is the expected value if the agent follows the policy $\pi$. Alternatively, we also often define action-value function $Q^{\pi}(s,a)$ as the expected reward when starting from state $s$ and selecting the action $a$ and then following policy $\pi$:

    % \[
    %   Q^{\pi}(s,a)=E_{\pi}\left\{\sum_{k=0}^{\inf} \gamma^k r_{t+k+1} | s_t = s, a_t = a \right\}
    % \]

    The main RL method applied to robotics is temporal-difference (TD) learning~\parencite{Sutton1988, Bradtke1996}. Based on the principles of TD learning, two major algorithms have been developed: on-policy SARSA (State-Action-Reward-State-Action) and off-policy Q-learning~\parencite{Watkins1989}. Additionally, most RL techniques have theoretical proofs of convergence~\footnote{Please note that what we have presented here is only a crude summary of RL in order to give sufficient context to the rest of our discussion. We point those interested by the subject to more exhaustive literature~\parencite{Sutton1998, Deisenroth2011}}~\parencite{Panait2005}.

    % Est-ce qu'il y a besoin d'un peu mieux expliquer où ça servirait à rien ?

    In the case of learning for multiple robots, the process is more challenging. In particular, other robots are often expected to be learning at the same time. At the very least, the learning process must take into account the presence of other dynamic agents. Yet the theoretical foundations behind MDPs rest upon the assumption that the environment is stationary~\parencite{Littman1994, Parker2008}. Consequently, adapting RL methods to multiple robots is not trivial. However, there is an extensive literature on learning in multiagent systems (MAS), of which some can be applied to MRS~\parencite{Stone2000, Yang2005, Panait2005}. In particular in the case of distributed control, concurrent learning, where each individual is an independant learner, has been widely studied. To that end, the framework of Dec-POMDP (for Decentralized Partially Observable Markov Decision Process) has been of interest in the MAS community~\parencite{Bernstein2002, Amato2013}. This model is a decentralized extension of basic POMDP~\parencite{Astrom1965}, which deals with real world partial observability. However, the optimal resolution of a Dec-POMDP (i.e. finding an optimal joint policy) was proven to be NEXP-complete~\parencite{Bernstein2002} and thus intractable. As such, most approaches rely on approximations of the model in order to solve the problem~\parencite{Beynier2011, Amato2013}. For instance, Seuken \& Zilberstein~\parencite{Seuken2007} have combined top-down heuristics with bottom-up dynamic programming to ensure linear complexity w.r.t. horizon length. Called memory-bounded dynamic programming, this method uses heuristics in order to limit the number of agents policies generated by dynamic programming. Dinbangoye et al.~\parencite{Dibangoye2015} assumed that interactions between agents took place locally in order to exploit the separability of the value function. This way they could transform a Dec-POMDP into a MDP to significantly gain in scalability. They thus could solve a task between up to fifteen agents while preserving convergence to an optimal solution.

    % In MAS, two sorts of learning can be used: \emph{team learning} or \emph{concurrent learning}~\parencite{Panait2005}. In the case of team learning, a single learner is used to learn the behaviours of the whole team. This does not mean that the team is necessarily homogeneous. In this case, as there is only a single learner, it is easier to transfer techniques and algorithms from classical RL. However, the fact that there are multiple individuals implies that the learning process faces a curse of dimensionality: the size of the states space increases with the number of agents. This also means that learning will be centralized which raises an issue in the case of fully distributed MRS. 

    % In comparison, in concurrent learning every individual is an independent learner. As previously said, this means that because the environment is not stationary, Markov's law is violated. In consequence, new learning techniques must be designed for multiple agents. In particular, concurrent learning is faced with an additional complex challenge: \emph{credit assignment}, i.e. how to divide rewards. Because each individual learns independantly, how to efficiently distribute a reward which depends on the collective performance is not easy. There are mainly two different ways to address credit assignment. On the one hand, it is possible to simply divide equally the team reward between every individuals, which is known as global reward. This means that even if there are vast differences between individuals' performance, all will be equally rewarded. Consequently, this may slow down the learning process as agents may not have sufficiently individually tailored feedback to improve on their strategy~\parencite{Wolpert2001}. On the other hand, it is possible to adopt local rewards, where agents are rewarded based on their individual performance. However, this do not ensure that the individuals will cooperate and may lead to the emergence of selfish behaviours. Some argue that there may not exist a definite solution to assign credit~\parencite{Balch1999}. Concurrent learning also has to deal with the learning dynamics caused by multiple learners. In particular, other individuals are not merely dynamic objects but are also co-adaptating at the same time. This means that for a given learner, other individuals adapt to this learner. In return, the learner will adapt in reaction to the factthat others adapted too and so forth. This raises similar issues to what is studied in game theory and in evolutionay game theory in particular~\parencite{MaynardSmith1973, Fudenberg1998, Bloembergen2015}~\footnote{As a reminder, evolutionary game theory was previously described in Chapter~\ref{chapter:model}. In consequence, it will only be mentioned here.}. Consequently, some have been interested in the research of Nash equilibria in joint learning in MAS so as to find the best-response policies of all agents. In particular, there has been several studies on trying to apply Q-learning techniques in order to find optimal policies in stochastic games~\parencite{Littman1994, Claus1998, Bowling2003, Greenwald2005, Kapetanakis2005}.

    However, as big as the literature on learning in MAS is, transferring reinforcement learning techniques from MAS to MRS still represents a challenge~\parencite{Yang2005}. In particular, while results in MAS offer interesting perspectives, MRS necessitate continuous actions and/or states spaces. This is something which is not that much studied in classical MAS. To overcome these problems, and the issue of continuous spaces in particular, several different solutions based on approximations have been proposed. For example Matarić proposed to extract the features from the learning space by reformulating states and actions into conditions and behaviours~\parencite{Mataric1997}. This way, the size of the spaces was greatly decreased. She also implemented shaping (i.e. decomposing a complex task into several simpler subtasks which are then learned in succession) in order to ease the learning process. In comparison, Fern\'{a}ndez and colleagues~\parencite{Fernandez2005} developed a learning MRS by discretizing the states space and then applying an algorithm to generalize from this discrete space. This particular algorithm, called ENNC-QL, is based on a supervised approximation of the value function. In the case of an adversarial MRS learning for soccer, Bowling \& Veloso~\parencite{Bowling2003} introduced GraWoLF (for Gradient-based WoLF). They used a policy gradient technique, which was proposed to overcome intractable and continuous states spaces~\parencite{Sutton2000} as well as WoLF (Win or Learn Fast), an algorithm to ensure convergence in the context of concurrent learning. Lastly, Stone \& Sutton~\parencite{Stone2001} also proposed a reinforcement learning method for a soccer competition. They implemented a SMDP (Semi-Markov Decision Process) Sarsa(\(\lambda\)) with linear tile-coding function approximation. With this method, they were able to have robots learn in a keepaway task.
     
    % Lastly, some have been interested in applying fuzzy logic (i.e. formal logic where truth values can take any real value between $0$ and $1$) to multirobot reinforcement learning. In particular, Gultekin \& Arslan~\parencite{Gultekin2002} proposed a modular-fuzzy algorithm with Q-learning where fuzzy sets were used to abstract the states and actions spaces. Furthermore, an internal model of the agents was built and used to estimate the action of each individual. 

    % In conclusion, a copious amount of work has been dedicated to applying reinforcement learning techniques for MRS. However the complexity and dimension of MRS implies that RL needs to rely on approximations and make critical assumptions about the world in order to cope with these challenges~\parencite{Yang2005, Parker2008}. This means that, while RL may be appropriate for single robot learning, MRS could benefit from a different framework to achieve automatic design. This is for these reasons that what we are interested in here is evolving robot control.

    %  by converting intermittent feedback into a continuous signal (à propos Mataric and shaping) -> wat

\section{Evolutionary Design for Distributed Robotics}

  \subsection{Evolutionary Robotics} % TODO: bof comme titre

    In this thesis, we study a different technique for the automatic design of robots: evolutionary robotics (ER)~\parencite{Nolfi2000, Doncieux2015a}. As we already covered some of the more technical details of ER in the Introduction, we are more interested here in briefly reviewing the contributions to the field of distributed MRS and highlight the challenges that are associated with. The key idea behind ER is to apply concepts of evolutionary computation to the design of robots. This means implementing concepts of selection and variation to build robust and adaptable robots. We previously showed that several design techniques have often been inspired in part by biology. For example reactive controllers are inspired by the concept of stimulus-response~\parencite{Brooks1986}, swarm robotics took inspiration from the collective behaviours of eusocial insects~\parencite{Bonabeau1999} and it is sometimes argued that major advancements in reinforcement learning mimic natural cognitive processes~\parencite{Montague1996}. Thus there is an interest in taking inspiration from evolution for the design of complex machines. In particular, ER uses evolution to approach robotic design in an holistic manner. The robot is considered as a whole and the evolution of its behaviour results from the interactions with the environment (and the other individuals in the case of MRS): ER works on embodied agents. Additionally, the evolutionary process is used as a meta-heuristic to search through the space of candidate solutions. In particular, ER works well in open environment because it can be used as a black-box optimization technique thanks to a loose formulation of the objective function. As such a minimum set of assumptions have to been made when using ER to design a robot~\parencite{Bongard2013a}. The major open issues in ER are on the transferability to real robots (i.e. reality gap)~\parencite{Mouret2012b, Cully2015}, the genotype and phenotype encodings (e.g. evolution of neural network topology~\parencite{Stanley2002}) and the selective pressures applied by the evolutionary algorithm~\parencite{Lehman2011, Mouret2012a}.

    At its core, ER can be considered as a learning technique. However, it may be troublesome to classify ER among learning algorithms. Indeed, while ER is a learning process in the machine learning sense of the word (i.e. a process which improves and optimizes candidate solutions according to a certain goal), it is not the case in a more biological sense: evolution is a phylogenetic adaptation while learning is an ontogenetic adaptation. This difference is even more critical now that combining evolution and learning represents an open issue in the field~\parencite{Urzelai2001, Mouret2014, Doncieux2015a}. Here we thus are careful to use the latter (i.e. biological) definition of learning and refer more precisely to reinforcement learning in this case. It is also important to note that ER and RL share several similarities~\parencite{Whiteson2012, Stulp2013, Doncieux2015a}. In both frameworks, the goal is for a robot to evolve (or learn) a behaviour (which is akin to a policy in RL) which maximizes a particular value: rewards in RL or fitness in ER. In particular, we can compare ER to a direct policy search in RL~\parencite{Kober2013} because it does not focus on finding an estimation of the value function of the states and actions but exploits the global value (i.e. fitness) of a policy. However, ER often necessitates higher computational time to find a good solution while dynamic programming in RL is guaranteed to find an optimal policy in polynomial time~\parencite{Littman1994, Whiteson2012}. In comparison ER works very well under partial observability and with problems that would require continuous or a large number of states in RL (as is the case for distributed MRS). In particular, ER explores the space of behaviours rather than that of states~\parencite{Panait2005}.

    % However, ER has also several advantages over reinforcement learning. First, ER methods work very well under partial observability. In particular, they are not constrained by Markov's law (see previous Section) in finding a good solution. While the field of RL is also concerned with partially observable markov decision processes~\parencite{Jaakkola1994}, this implies that ER may be more suitable for the design of robot control in the face of uncertain environments. ER is also more appropriate in dealing with problems that would require continuous or a large number of states in RL (as it is the case for MRS). Again, as previously discussed, RL may be able to deal with this kind of problems. However, it implies that the problem's definition must be approximated before more classical RL algorithms can be applied. In comparison, ER explores the space of behaviours rather than that of states, which makes it a better solution when there is an explosion of the size of the states spaces~\parencite{Panait2005}. Finally, evolutionary robotics does not require to build a complex representation or a state-action space because it evolves its own representation. It is interesting to note that, because of the advantages of ER when compared to RL, there is also a copious literature (which we only mention here) on trying to include certain concepts of evolutionary computation to reinforcement learning~\parencite{Whiteson2012}.

    ER can also be used as an online design method. In the "classical" framework of evolutionary robotics, the evolutionary algorithm is called offline. This implies that there are two distinct phases in the development of a robot: the design phase (i.e. evolution of controllers) and the operational phase (i.e. deployment of robots)~\parencite{Doncieux2015a, Francesca2016}. This thus is based on the assumption that the environment where the robots are deployed is the same that the one where they were evoled. Or at least it considers that the evolved controller will be capable of adapting to the new environmental conditions. In an online method the design process is done directly in the operation environment. In the case of multiple robots, this gave rise to distributed online evolutionary robotics, often called \emph{embodied evolution}~\parencite{Ficici1999, Watson2002}. Because of the complexity of learning exact policies with multiple robots, the field of online evolution has sparked greater interest in multirobot settings than with a single robot~\parencite{Doncieux2015a}.

    % We already mentioned the work of Montanier \& Bredeche~\parencite{Montanier2011, Montanier2013} in Chapter~\ref{chapter:model} who, based on this framework, studied the evolution of altruistic cooperation. Most notably, they observed the evolution of altruism in a of tragedy of commons situations~\parencite{Hardin1968} and its relation with genetic relatedness and dispersion. While this work is at the frontier between model and design, it can clearly give insight into the development of cooperative robots in embodied evolution.

  \subsection{Evolving Collective Robots}

    While ER has been mainly focused on the design of single robots~\parencite{Nolfi2000, Doncieux2015a}~\footnote{As the focus of this manuscript is on multirobot systems, we will not discuss the literature on the subject of ER for single robots. Interested readers should direct their attention towards more extensive reviews of the field~\parencite{Floreano2008, Bongard2013a, Trianni2014b, Doncieux2015a}.}, its potential for the engineering of complex collective systems is well known~\parencite{Baldassarre2003}. For instance, Reynolds~\parencite{Reynolds1992} proposed an evolved version of its "boids" simulation. He evolved a herd of between 16 and 20 critters whose goal was to avoid both obstacles and a predator. He used genetic programming where the evolved programs (i.e. genotypes) were Lisp expressions based on simple behavioural functions: turn, look-for-obstacle, look-for-friend and look-for-predator. He showed that he could evolve vision-based coordinated motion for a herd that avoided collisions.

    More generally, evolution has been widely used in the context of swarms~\parencite{Brambilla2012, Francesca2016}. In particular, it allows to divert from the classical approach of manually designing individual behavioural rules. On the contrary, ER can automatically evolve self-organized control according to a group-level fitness score. Evolution was used in the context of the Swarm-bot project~\parencite{Mondada2005}, where the goal was to engineer a swarm of simple identical robots capable of using self-assembly to navigate accross rough terrain and achieve different collective tasks. In particular, Baldassarre and colleagues~\parencite{Baldassarre2003b, Baldassarre2007} achieved coordinated motion between a swarm of $36$ simulated robots. Because the robots were connected to each other in line, they had to coordinate their movement in order for the whole swarm to reach the objective. Robots were controlled by a neural network which, given the traction on the robot in $4$ directions, computed the desired motion. They showed that the evolved robots were also capable of high adaptability and generalization under various environmental conditions: number of robots, shape of the swarm, variation in the rigidity of the robots' connections, rough terrain and robots connected through a passive object. The controllers were also successfully transferred on real robots. The swarm-bot framework was also investigated for collective transport of objects by Groß and Dorigo~\parencite{Gross2004a}. They evolved the neural networks of up to $16$ autonomous robots that were capable of collectively push or pull an objet towards a moving target. Trianni et al. evolved a swarm of robots displaying aggregation behaviours~\parencite{Trianni2003} and that later cooperatively navigated an environment to overcome their limited sensory capabilities and avoid falling into holes~\parencite{Trianni2004}. In the context of swarms, Duarte and colleagues~\parencite{Duarte2016} developed $10$ simple and small real robotic boats that they evolved for basic tasks: homing, dispersion, clustering and area monitoring. They used artificial neural networks evolved with NEAT~\parencite{Stanley2002}. The best $3$ controllers of each evolutionary run were then transferred into real robots to test them. Finally, Hauert et al.~\parencite{Hauert2009} evolved a group of $20$ simulated flying robots, or MAVs (Micro Air Vehicles), tasked with the establishment of a communication network. Launched from a human rescuer, the robots had to coordinate to find the other rescuer and then set and maintain a multi-hop communication link between these two rescuers. Every robot was controlled by a neural network which outputted the turn rate of the MAV given the heading compass of the robot and the number of network hops that separate it from the two rescuers. The connection weights of the network were encoded in a binary string and evolved by a genetic algorithm.

    Additionally, D'Ambrosio and Stanley have been interested in using HyperNEAT~\parencite{Stanley2009a} to evolve teams of agents capable of coordination. For instance, they implemented~\parencite{DAmbrosio2008} HyperNEAT in a predator-prey experiment. In particular, they evolved neurocontrollers for predators' behaviours which had to hunt moving prey. The predators could not see other predators and thus had to learn complementary roles so that they would not interfere with each other. They used Computational Pattern Producing Networks (CPPNs) which allowed for the agents to assign roles depending on their relative geometry. Furthermore, they showed that seeding evolution with the genome of a single pre-evolved agent could benefit the learning process by injecting domain knowledge. D'Ambrosio and colleagues~\parencite{DAmbrosio2012} investigated coordination between $4$ robots thanks to direct neural network communication. More precisely, the neural network of each robot was connected to the internal nodes of the networks of other agents, which they called the \emph{hive brain}. Robots could simply move left or right and the task was for them to synchronize their motion (akin to several pendula) with no sensory information about other robots. They showed that the agents could evolve an efficient communication strategy that led them to synchronize their motion. Moreover, the evolved controllers could transfer well to real robots.

    % On the subject of coordinated motion, Quinn et al.~\parencite{Quinn2003} were among the first to evolve controllers for a cooperative task in physical robots. While they used three robots and as such it may be far-stretched to consider this swarm behaviour, their work had the main features of a swarm: homogeneous, distributed and emergence of a collective behaviour. Their robots were capable to perform formation-movement as well as adopting distinct roles between each others. Furthermore, the robotic agents were equipped with very limited sensory capabilities. More recently, Hauert and colleagues~\parencite{Hauert2014} evolved a group of twenty simulated flying robots with the task of establishing a communication network. More precisely, they designed a task where a rescuer launches twenty autonomous robots from his position. The robots have then to coordinate to find the other rescuer and set a communication link between the two rescuers that has to be maintained for up to thirty minutes. As the communication range of the robots is way shorter than the distance between the two rescuers, they have to cooperate to create a communication link. 

    %In particular, there has been different works both on the competitive and cooperative side of social behaviours. On the competitive side, most of the work is focused on competitive co-evolution~\parencite{Floreano1998, Floreano2008}. The biological inspiration behind competitive co-evolution is that several species (e.g. two in the simplest models) will be in competition for survival. This means that changes in one particular species may lead to adaptative change in the other species which again may lead the former species to adapt to these changes. In the context of evolutionary robotics, competitive co-evolution can lead to competitive improvements where both "species" are incrementally improving their behaviour in response to the improvements of that of the other species. This incremental process can give the possiblity to shape the adaptation of the robots towards more and more complexity without having to specifically design the fitness or the selection process to that end. The classical example of competitive co-evolution is that of the predator-prey model~\parencite{Floreano1997}. In this model, inspired by the dynamics of the Lotka-Volterra predator-prey equations~\parencite{Yorke1973}, a species of predators and a species of prey are co-evolved in the same environment. In consequence, the prey as to adapt to the predator's strategy in order to escape it and the predator has to adapt to that of the prey to be able to go on catching prey. Nolfi \& Floreano studied this co-evolution problem in ER and showed that the best evolved performance for a predator strategies was the one obtained when co-evolving both the prey and the predators~\parencite{Nolfi1998}. However, while these give really interisting pratical and theorical insights on the evolution of complex behaviours, work on competitive co-evolution are still few. 

    % Some have also been interested in the evolution of specialisation (or division of labour). Specialisation means that the individuals divide between several roles, either to achieve a task more efficiently or to achieve multiple tasks at the same time. For example, Ferrante and colleagues~\parencite{Ferrante2015} have investigated the behaviours of leafcutter ants in a task were robots had to bring leaves into the nest. More precisely, robots could either adopt a specialist or generalist behaviour. When specialised, some robots cut the leaves and put them into a temporary storage area and others take the leaves from the storage area to get them to the nest. Generalists would do both. They showed that they could evolve division of labour when environmental conditions (the presence of a slope) made the presence of specialists more interesting.

    % Là à la fin il faudra que je recase mon paragraphe avec homogeneous etc... pour parler des open challenges en gros

    % Ca fait peu quand même niveau gens qui s'intéressent à division of labour. Floreano met le truc de Danesh avec les fourmis en ER. A investiguer.

\section{Genetic Team Composition and the Evolution of Cooperation}

  \subsection{Team Composition and Levels of Selection}

    While ER is often viewed as a black-box optimization framework, several critical design decisions impact its efficiency when applied to robotics~\parencite{Trianni2014b}. As such, these design choices represent a specific problem in the field of evolutionary robotics~\parencite{Doncieux2014a}. We are interested here in the evolution of cooperative robots in ER. In that context, two features of the evolutionary algorithm are especially critical~\parencite{Waibel2009, Lichocki2013}:

    \begin{description}
      \item[Level of selection] {This represents the level at which selection is applied, which is in part impacted by the way fitness is distributed between individuals.}

      \item[Team composition] {This is the genetic composition of the team of robots, which corresponds to the manner in which robots from a group are encoded given the population of evolved genotypes.}
    \end{description}

    Selection can act at the level of the \emph{group} or that of the \emph{individual}. Namely, the level of selection is concerned with the way fitness is attributed to each individual in the team. If group-level (or team-level) selection is used, every individual is equally rewarded by the team's performance in the task. This means that in the case where every individual separately evolves, the evolutionary process may be slowed down by the fact that the performance feedback may not be adequatly tailored to the individual. In comparison, under individual-level selection, each individual is rewarded based on its own performance. This in turn means that the emergence of cooperative behaviours is not ensured as the individuals could benefit from selfish actions. This problem is often known in multiagent learning as "credit assignment".

    On the other hand, team composition is a well-known design choice for MRS in general (as we previously talked about in Section~\ref{sec:MRS}). Here we focus solely on the issue of team composition with relation to the control of robots, regardless of morphology. In evolutionary robotics, every individual in an homogeneous team is composed of the same genotype. This approach is thus often called a clonal approach (as agents are clones of each other). In comparison in an heterogeneous group every individual is encoded by a different genotype, a process also known as an aclonal approach.

    Homogeneity tends to facilitate the maintenance of novel beneficial mutations that could be lost in an heterogeneous context~\parencite{Quinn2001}. As such, using homogeneous teams can lead to finding solutions in less computational time. Also, because the performance of one individual (w.r.t. fitness score) is the same as every other agent in the team, it should be easier to evolve cooperative solutions. Indeed, an individual can benefit from its behaviour as soon as it benefits the whole group~\footnote{Given the strong ties of this thesis with biology, it is interesting to note that this process is similar to the evolution of altruism under kin selection mechanism (which requires genetic relatedness between individuals).}. Additionally, because every individual has the same control, it is easier to achieve coordination when it is expected that the agents behave similarly. However heterogeneity implies that individuals are different and as such may rely on diverse capabilities. In particular, when it is expected that the agents work together in complementary ways (e.g. division of labour), this behavioural asymmetry is easier to achieve with heterogeneous individuals. However an issue in heterogeneous teams is that, depending on the level of selection, the evolution of cooperative solutions may be hindered by the exploitation of selfish behaviours. Indeed, if each individual can selfishly benefit from its behaviour, it may not contribute to the collective action. The main approaches that have been adopted in ER have been centered on using team-level selection with homogeneous teams or individual-level selection with heterogeneous teams~\parencite{Waibel2009}.

  \subsection{Team Composition in Evolutionary Robotics} % Bof

    When evolving cooperation is concerned, the classical approach is to use homogeneous teams of individuals (or clonal approaches). In particular, a large part of the literature is concerned with swarms, which most often are constituted of homogeneous individuals. For instance, the works in swarm behaviours presented in the previous Section all evolved a single population of genotypes where each genotype encoded for every agent in the group. As we previously explained, this can be explained by the fact that homogeneous teams are a good choice when evolving cooperation and strongly coordinated behaviours. Waibel and colleagues~\parencite{Waibel2009} gave an experimental proof of this assertion in a work dedicated to this issue. They produced a study on the influence of team composition (homogeneous or heterogeneous) and level of selection (team-level or individual-level) in $3$ different foraging tasks that did not require specialisation: an individual one, a cooperative one and an altruistic one (i.e. that required individuals to pay a cost when cooperating). The manner in which they evaluated or selected the individuals depended on the exact combination of team composition and level of selection. They showed that, when cooperation was needed, an homogeneous team of individuals under group selection was the best performing setting. But others have also used homogeneous teams in tasks that may require heterogeneous behaviours. For example, division of labour can be achieved between homogeneous individuals if the agents specialise during their lifetime thanks to varying initial conditions, development or environmental cues. In that context, Ferrante and colleagues~\parencite{Ferrante2015} studied task partitioning, i.e. where different tasks have to be done in sequence, in the context of an evolved population of simulated foragers. They investigated the evolution of generalist behaviours (i.e. individuals who carry every task) and specialist behaviours. The teams of robots were homogeneous and they simulated evolution both with pre-evolved building blocks (i.e. pre-adapted basic behaviours) and de-novo starting only from low-level behavioural primitives. They showed that specialisation could evolve based on environmental information which was used by the agents to dynamically assign roles. Additionally, they demonstrated that particular environmental conditions, in their case a slope which made task partitioning more useful, could affect the evolution of specialists. Indeed, this slope could be used to facilitate transport and also decrease the cost of switching from one role to the other.

    A particular manner in which heterogeneity is sometimes studied, especially in the field of multiagent learning, is by using Cooperative CoEvolutionary Algorithms (CCEAs)~\parencite{Potter1994}. Originally, CCEAs are used to search solutions to a given problem by decomposing this problem into subcomponents and have different populations concurrently search for a solution for each subproblem. In the context of ER, the principle is that a team is composed of robots whose controllers are separately evolved in different populations. This approach is loosely inspired by the biological coevolution of multiple species. This method is particularly useful when trying to evolve highly specialised individuals. Blumenthal \& Parker~\parencite{Blumenthal2004} used a CCEA to evolve four differently abled hexapod predators in a predator-prey scenario. The issue was to ensure that the predators would evolve a correct behaviour given their own movement capabilities to prevent the prey from escaping. They used a genetic algorithm based on punctuated learning, where the learning system is updated only after a given number of generations, and showed they could successfully evolve the predators to capture the prey. Each individual was controlled by a neural network evolved in a different population. Similarly, Yong \& Miikkulainen~\parencite{Yong2009} evolved $3$ agents in predator-prey scenario where the prey is faster than the predators. They used a method of neuro evolution, enforced subpopulations, which is a cooperative coevolution method where several populations of hidden neurons are separately evolved. Each of the $3$ predators was encoded by neurons from a different subpopulation. The authors showed that the individuals evolved by cooperative coevolution outperformed a single centralized controller on both efficiency and robustness. Lastly, Nitschke and colleagues~\parencite{Nitschke2012} introduced Collective Neuro-Evolution (CONE), a cooperative coevolutionary algorithm for the evolution of artificial neural networks, in order to evolve specialists. They evolved recurrent feed-forward networks for a team of between $50$ and $100$ robots, where the genotype of each robot is separately evolved. The robots had to collect blocks of different types and place them in sequence in a designated area. CONE implemented two metrics to control for specialisation and genotype similarity in order to efficiently apply recombination between populations. The goal was thus not to lose specialists during recombination. Therefore, coevolution can be used to efficiently evolve heterogeneous behaviours. However one main issue with this method is that it relies on evolving a population for each of the expected behaviours. In consequence, it needs \emph{a priori} knowledge on the task at hand in order to structure the population accordingly. Additonally, as population size is critical in the evolutionary process, evolving several populations at the same time may be costly.

    In comparison, few works have been interested in using an heterogeneous team of individuals evolved from a single population. The exact advantages of heterogeneous approaches are still not clear in the literature. Quinn~\parencite{Quinn2001} produced a comparison of homogeneous and heterogeneous approaches in a task which required two individuals to adopt coordinated motion. He compared a clonal approach, where both individuals come from the same genotype (i.e. homogeneous team), and an aclonal approach, where individuals come from different genotypes (i.e. heterogeneous team). Coordination was achieved when the individuals moved from their starting positions while staying close to one another without collision occurring. Individuals needed to take specific roles in order to coordinate efficiently. He showed that the aclonal approach outperformed the clonal approach in this task. However, Tuci \& Trianni~\parencite{Tuci2014} then ran a similar study and found different results. More precisely, they evolved a team of two robots in a setting where one of them had to stay in a designated area (the nest) and another had to move back and forth between the nest and foraging sites. As such specialisation was explicitly required to carry the task and the roles were clearly distinct. In this case, they showed that this time the clonal approach clearly outperformed the aclonal one in both efficiency and robustness.  Therefore, the exact impact of genetic team composition on the evolution of cooperation is still an open issue. In particular, the manner in which specialisation is affected by homogeneous or heterogeneous teams is still debatable. The evolution of heterogeneous behaviours does not require heterogeneous control. However, it then requires that the individuals have the capabilities to dynamically specialise. In the case of Tuci and Trianni, this was achieved thanks to a continuous time recurrent neural network. However, it may not always be desired nor cheap to endow agents with those specific capabilities depending on the context. With no \emph{a priori} knowledge of the environment, it could be preferable to design the evolutionary process so that it may evolve efficient behaviours in simple robots. Additionally, it is not always known beforehand whether heterogeneous behaviours would even be advantageous. In conclusion, there is the need for additional works on team composition with regard to the evolution of cooperative behaviours.


\section{Conclusions}

  Our goal in this Chapter was to give a brief presentation of multirobot systems and to discuss the different methods with which they could be designed. Furthermore, we wanted to highlight a critical design choice that we will explore in the two following Chapters. We showed that the design of distributed MRS, as for robotics in general, is complex and may be facilitated by resorting to automatic design. However, this is not an easy task. In particular, machine learning methods (i.e. RL) which may work for single robots do not cope well with the complexity of MRS and require to approximate the problem at hand so that it can be dealt with. In comparison, ER is another possible method which functions as a black-box optimization and thus may function well in open environment. As such we are interested in studying an open question in the field of ER: the influence of genetic team composition in the evolution of cooperative robots.

  % Faudra insister sur le fait que notre premier papier c'est une étude comparative à la Quinn et Trianni

  In the next two Chapters, we are interested in the evolution of cooperation among a group of heterogeneous robots. We focus on the nature of the coordination strategies evolved. Thus we consider teams of two genetically unrelated robots. While these two robots are morphologically identical, the heterogeneity in their control raises the issue of evolving cooperation when selfish behaviours can emerge. From this stems a tradeoff between using heterogeneity to evolve efficient coordination strategies and the challenge of evolving cooperation. This issue is discussed in the first Chapter. In the second Chapter, we focus solely on evolving a particular type of coordination: division of labour. Because we want this strategy to appear among heterogeneous individuals with no added capabilities, we study the issue of achieving genotypic polymorphism. Namely, we want multiple different behaviours encoded by different genotypes to coexist at the same time in a single population. In both studies, we do not use real robots. We consider our contributions here to be mainly theoretical and act as general design concepts regardless of a specific robot model. Furthermore, the transferability of evolved behaviours on real robots as well as the online evolution on physical robots are both major challenges~\parencite{Floreano2008, Doncieux2015a} in ER that would represent a separately substantial study for this thesis. We thus consider these issues to be beyond the scope of our work.

  % To conclude this introduction to this part of the manuscript, we want to clarify that, as in the first part of the thesis, we do not use real robots. We consider our contributions here to be mainly theoretical and act as general design concepts for others in the field. In particular, our robotic agents and settings are pretty simple so that we do not make much assumptions that would relate only to our study. Additionally, applying ER on real robots is still considered to be a challenge in the field~\parencite{Floreano2008, Doncieux2015a}. We believe that the recent developments aiming at either simplying experiments on physical robots or increasing transferability between simulation and reality to be beyond the scope of this here thesis. 

  % Faudra dire en quoi on est différent de Potter2001 sur le premier papier (c'est du compétitif, c'est la seule différence ?)

  % !!! Quinn a dit c'est peut-être bien l'aclonal !!!

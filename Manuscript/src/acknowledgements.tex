\chapter*{Acknowledgements}
\addstarredchapter{Acknowledgements}
\chaptermark{Acknowledgements}
% Trick required for Acknowledgements to be seen in TOC

%%%%%%%%%%%%%%%%%%%%%%%%%%%%%%%%%%%%%%%%%%%%%%%%%%%%

\epigraph{\textit{\hfill What deadline ?}}{--- \textup{Arthur Bernard}}

A thesis can hardly be considered as the work of a single individual. It is dependent on collaborations, discussions and encouragements from many different people. As such I would like to express them my deepest gratitude.

First, I would like to thank Guillaume Beslon, Aurélie Beynier, Pierre-Henri Gouyon, Nicolas Maudet and Richard Watson for having accepted to be part of my thesis committee. This represents a substantial amount of work and I am very grateful that they could take some time to read my manuscript and attend my defense. Additionally I want to thank Guillaume and Richard for reviewing the manuscript and providing very helpful comments and corrections.

I want to thank in particular my two supervisors Jean-Baptiste Andre and Nicolas Bredeche without whom I would not be writing these acknowledgements right now. It may sound cheesy to say it but I think I could not have ask for better advisors. They knew exactly when to encourage or push me and really made sure that this thesis could go as smoothly as possible. This was a pleasure to discuss and work with them and I consider most of this work to be as much mine as it is theirs. I must especially thank Nicolas for his active presence, helpful supervision and frequent interactions. It was a pleasure to discuss science as much as video games and music. I have also Nicolas to thank for getting me into research and providing me with helpful directions when I was thinking about doing a PhD. Thanks to him, I was able to leave my job for a more stressful and less paying experience. I am also grateful to Jean-Baptiste for teaching me most of what I now know about evolutionary biology. He was often enthusiastic and very optimistic about the results of this work which was helpful in the inevitable moments of doubt that occurred during these 3 years.

I had the chance to work in an exceptional environment at ISIR (despite the summer heat). I want to thank all of the wonderful people I met there and with which it was a pleasure to interact. I want to address special thanks to Benoît, Jean-Baptiste, Medhi and Stéphane for their general kindness and interesting discussions beyond scientific matters. I am grateful to my colleagues with whom I shared the J01 and 310 offices. Those offices were messy, noisy and at times even dangerous because of our overall lack of skill at throwing darts. This thus was the perfect working environment. I want first to thank the Ancient Ones, the ones that were finishing their PhD as I was only starting mine: Charles, Charles (the other one), Florian and Jean. I would have loved spending more time with you all. Then I thank those with whom I spent most of my time here and who made this experience a lot more pleasant (most of the time). First many thanks to Alain, even if I'm not sure to which extent you were really helpful for the productivity of the office. Anyway, your frequent (some would say constant) breaks in our office would always lighten up the mood and give the opportunity to pull a prank on someone. Thanks to Antoine for the pleasant discussions, choosing to take our (not so many) complaints to the team meetings and organizing the friday breakfasts. I want to thank Carlos for following in Antoine's footsteps and trying to bring order in our chaos. And also for singing in a way that could often be mistaken for satanic prayers. Many thanks to Erwan for being cheerful (most of the time anyway), providing a safe space for the room's penguin and startling everyone with his frequent but brief fits of rage. I am grateful to François for being one of the calm ones, bringing some sanity to the room and stealing my first place at Duolingo. Many thanks to Ghanim for making sure that a fresh batch of coffee would always await us in the morning. Thank you Guillaume for acting as the token hipster and racing me towards the end of the PhD. Not so thankful for stealing potential listeners from my defense. Thank you Leni for being so pleasant to talk to and for always providing useless but funny inputs to many discussions. Many thanks to Nassim for being comically grumpy all the time and with whom I shared so many dart games (and mostly lost). I am also grateful that you took most of the burden of organizing the journal club. Thank you Pierre for transforming any possible conversation into a vigorous debate, being so pleasant and for very interesting discussions (in majority about Overwatch). Many thanks to Thibaud for his constant spam of articles on which we could have dedicated a whole PhD and for always following Pierre in his debates with so much passion (and noise). Finally I want to thank all (at least those that I can remember of, sorry for the others) the interns that I met during these 3 years : Antonin, Daphné, Émilie, Élias, Florian (whose moustache shined as bright as his humour), Omar, Quentin, Yann. While I shamelessly copied Florian (whom I thank once again) for his long list of personnalised acknowledgments, this list is far from exhaustive. Thank you to all of you that shared my stress, asked me daily if I wanted to have lunch (which I still do not) and for the many many games of Counter Strike, Nexuiz, Trackmania and darts.

This PhD would not have been possible without those with which I shared my life outside of the office. First I am very grateful to Lindsay who made the uninformed mistake of dating someone at the beginning of his PhD. She supported me at every turn, encouraged me when I needed it and was more than understanding when most of my time was stolen by my thesis. You rock. I am also greatly thankful to all of my friend for keeping my sanity at decent levels thanks to frequent bar crawls and board games nights : Brice, Diane, Drago, Fabien, Marc, Matthieu, Max, Pierre and Romain. You tolerated my constant rambling about hyenas and agreed not to ask too many questions about how my thesis was going and for that I am very grateful. Lastly, I want to thank the NewRetroWave YouTube channel for providing me with music to keep me going while I was writing this manuscript.

Finally I am thankful to my family. First many thanks to Guillaume without whom I would probably not have done a PhD in the first place. You always made sure that I knew I could count on you if I ever needed support during the many difficult times of a PhD and you were nice enough to keep me awake at night by screaming at your computer. Then I obviously want to thank my parents who supported me all the way since the beginning and are the reason that I am who I am today.


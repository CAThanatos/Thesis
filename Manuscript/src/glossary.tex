%%%%%%%%%%%%%%%%%%%%%%%%%%%%%%%%%%%%%%%%%%%%%%%%%%%%%%%%%%%%%%%%%%%%%
%% Examples on how to use index/glossary
%% /!\ Make sure to use make && make glossary (or make full)

%% Indexed entries
\newglossaryentry{entry}{
  name={entry},
  description={Entry's description.}
}
\newglossaryentry{yaentry}{
  name={yaentry},
  description={Entry's description.}
}
\newglossaryentry{snentry}{
  name={snentry},
  description={Entry's description.}
}
\newglossaryentry{smentry}{
  name={smentry},
  description={}
  % with an empty description, the term shows in the index but not the glossary
}

%% Acronyms (are also indexed)
\newacronym{ER}{ER}{Evolutionary Robotics}
\newacronym{MRS}{MRS}{Multirobot Systems}
\newacronym{IBM}{IBM}{Individual-Based Modeling}
\newacronym{EGT}{EGT}{Evolutionary Game Theory}
\newacronym{ESS}{ESS}{Evolutionarily Stable Strategy}
\newacronym{RL}{RL}{Reinforcement Learning}
\newacronym{MDP}{MDP}{Markov Decision Process}
\newacronym{Dec-POMDP}{Dec-POMDP}{Decentralized Partially Observable Markov Decision Process}

%% Easy way to use entries
\newcommand{\entry}{\gls{entry}\xspace}
\newcommand{\yaentry}{\gls{yaentry}\xspace}
\newcommand{\snentry}{\gls{snentry}\xspace}
\newcommand{\smentry}{\gls{smentry}\xspace}
\newcommand{\acro}{\gls{ACRO}\xspace}
\newcommand{\yaacro}{\gls{YAACRO}\xspace}

% \gls : singular form of entry
% \glspl : plural form of entry
% \Gls : singular form of entry (with Upper case first letter)
% \Glspl : plural form of entry (with Upper case first letter)

% \acrshort : print "<abbrv>" version of acronym
% \acrlong : print "<full>" version of acronym
% \acrfull : print both "<full> (abbrv)" versions of acronym

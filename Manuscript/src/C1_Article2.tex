\chapter{The optimization of collective actions by individual selection}
\label{chapter:C1_article2}

\minitoc[n] % minitoc without title

In this Chapter, we investigate the influence of the nature of coordination behaviours on the optimization of collective behaviours. This Chapter is presented as a draft for a journal article which will be submitted before the end of the year.

% \begin{quote}
%   \fullquote{Bernard2016a}
% \end{quote}

In the previous Chapter we showed that the evolution of collective actions was hindered by the evolution of coordination. Here we are more interested on the impact of coordination on the transition from different collective equilibria. More precisely, these behaviours allow to reap benefits that would not be obtained in a solitary manner (e.g. collective hunting). However, because they require coordination, it is not clear how these collective behaviours are reached. In particular, the need for multiple individuas to coordinate implies that several equilibria evolve may evolve. Because benefits are reaped through collective action, a single individual would not benefit by deviating from the evolved equilibrium. This means that mutant acting toward a different equilibrium could not be selected. This would be the case even if the equilibrium would be more advantageous for the group. In consequence, these collective equilibria are stable when evolved but the evolution a particular equilibrium is not straight foward. This raises the issue of the optimization of collective actions, i.e. the transition toward an optimal equilibrium when another collective equilibrium has already been evolved.

One classical mechanism to solve this issue is that group selection. Because those behaviours are beneficial at the level of the group, then this should mean that they are selected at the same level. However, we want to study if collective behaviours could be optimized by individual selection only. To that end, we choose to model the example of collective hunting. In particular, we use a similar model as the one in Chapter~\ref{chapter:C1_article1}. Namely, individuals are evolved in an environment where they can hunt two differently rewarding types of prey: \emph{boar} and \emph{stag}. Each type of prey corresponds to a different collective equilibrium: suboptimal for the boar and optimal for the stag. Our goal is thus to study the transition from the suboptimal equilibrium (i.e. boar hunting) to the optimal equilibrium (i.e. stag hunting).

We reveal that under simple ecological features (only two prey in the environment), the transition to the optimum is impossible. However, under more realistic assumptions where the individuals have to choose between multiple prey, then the optimal equilibrium is evolved in $8$ replications out of $30$. In particular, the individuals now have to coordinate in order to achieve cooperation. This means that they need to react to each other behaviour. In consequence, they react to a mutant's behaviour which may allow to reap the benefits of stag hunting. However, the coordination strategy observed in one where both individuals try to simultaneously choose the prey on which to hunt. They thus adopt very symmetrical behaviours.

We then increase the complexity of the neural networks to study the impact of the evolution of a more complex coordination strategy. We observe the evolution of a more efficient asymmetrical strategy where the individuals adopt two different roles: the \emph{leader/follower} strategy. In this case, we reveal that the transition to the optimum is facilitated as stag hunting evolves in $24$ replications out of $30$. In this strategy, only the leader chooses the prey and the follower goes on the same prey. In consequence, while previously choosing to hunt a stag was a collective decision making problem, it is now an individual problem. This means that a mutant leader going on a stag is now sufficient for both individuals to reap the benefits of stag hunting. Furthermore, the leader/follower strategy evolved because it was more efficient. Thus we showed that the evolution of an individually adaptive coordination strategy could lead to the optimization of a collective behaviour.
\chapter{Discussion}

% \epigraph{\textit{This is the end. My only friend, the end.}}{--- \textup{Jim Morrison}}

\minitoc[n] % minitoc without title

Bla bla bla on a utilisé l'ER pour étudier deux choses différentes etc...
Du coup on va discuter de chaque partie séparément

\section{Modeling the Evolution of Cooperation}

	\subsection{Contributions Synthesis} % TODO: check english

		% TODO: bien vérifier que tout est au passé (je crois que c'est pas super le cas)

		In the first part of this thesis we have been interested in the influence of coordination in the evolution of mutualistic cooperation. We focused in particular on the proximate explanations of coordination and their impact on the ultimate evolution of cooperation. Most studies centered on the question of the evolution of cooperation have often been dedicated to explaining the stability of altruistic behaviours. Mutualistic actions in comparison have usually been ignored. However, while mutually beneficial actions do not raise any issue of stability, the origin of this type of cooperative behaviours was not trivial. Because they require coordination, how can a cooperative behaviour spread from an initial population of solitary individuals represents an open question. We thus were interested in the boostrap of mutually beneficial cooperation.

		To that end, we use evolutionary robotics as our modeling framework. Because we are interested in the origin of cooperative actions rather than their stability, the modeling of mechanistic constraints is critical in order to get a full understanding. Namely, the convergence to a cooperative solution is impacted by the availability of mutations, i.e. the possibility for a particular mutant to appear in the population. While classical models in evolutionary biology are of great help in understanding the dynamics of the evolution of cooperation, they require to make critical assumptions with regards to the convergence of a cooperative solution. This is what has motivated our choice to use individual-based modeling and evolutionary robotics in particular as a modeling method. In particular, evolutionary robotics allow to model the mapping between genotype and phenotype and thus study the proximate mechanisms at play in the evolution of cooperation.

		In a first study, we focus on the impact of modeling the mechanics of behaviour in the evolution of mutualistic cooperation. To that end we take inspiration from the game theoretical model of the stag hunt and study the transition from a solitary equilibrium (hare hunting) to a cooperative equilibrium (stag hunting). We first reveal that there is a drastic difference between the results predicted by classical models of game theorey and what we observe in evolutionary robotics. With a classical model, the transition to stag hunting always occurs. In comparison, with a model in evolutionary robotics, this transition is nearly impossible ($1$ replicate out of $30$). Furthermore, we show that even under maximal genetic relatedness (i.e. individuals are clones of each other), the evolution of cooperation is still unlikely ($6$ replicates out of $30$). We thus reveal that the mechanistic constraints are critical for the origin of cooperation in the stag hunt game. The evolution of cooperation is faced with a chicken \& egg dilemma: for cooperation to be selected, it needs to beneficial. Yet the success of the cooperative action requires that others have evolved the capacity, which is not beneficial on its own. If we assume that a single mutation can lead to the evolution of cooperation then we consider that a same mutation is responsible for the both a modification of the preferred prey and the capacity to coordinate. We show here that doing as such hides the some of the issue (TODO: phrasing bof). Moreover we show in our simulations that the individuals need to evolve a complex behaviour in order to be able to coordinate. Therefore, it is necessary to take into account the mechanics of behaviours in order to fully understand the evolution of cooperation. This means that there is a need for complementary frameworks which model these mechanisms in the toolbox of evolutionary biology.

		In the second Chapter, we studied how the nature of coordination behaviours may impact the transition between collective equilibria. More precisely, we focused on the issue of selecting between multiple stable equilibria. When a collective equilibrium has emerged, no single mutant has a selective advantage to deviate from this equilibrium. As such, this raises the issue of the transition to the optimal equilibrium when another equilibrium already evolved. Our goal was to study if individual selection alone could lead to the optimization of group traits. To that end, we used a model of collective hunting where individuals can choose between two types of prey: boar (suboptimal prey) and stag (optimal prey). We revealed that the transition towards the optimal prey was impossible under simple environmental features (only one prey of each type). However, we revealed that those results are highly different when we make more realistic assumptions about collective hunting. Surprisingly, when the environment is more complex ($9$ prey of each type), the switch to stag hunting can sometimes occur (in $8$ replicates out of $30$). Under such environmental conditions, it is now necessary to collectively decide on which prey to hunt. This means that the evolution of coordination is now necessary to achieve cooperation. In consequence, each individual evolved the capacity to each to the other individual's behaviour. This means that a mutant can indirectly change the behaviour of the group and thus lead to stag hunting. However, we observed that the coordination strategy evolved is a very symmetrical one where both individuals adopt the same behaviour. We then added a duplication event which could randomly increase the neural complexity of individuals by duplicating their neural networks which could evolve separately. We showed that the transition to stag hunting was then highly facilitated ($24$ replicates out of $30$). Furthermore, we revealed that individual now evolved a strongly asymmetrical coordination strategy through specialization: a leader/follower strategy. In this strategy, the follower only reacts to the behaviour of the leader which it tries to follow. This means that a mutation on the leader is sufficiant to change the group's behaviour and thus reap the benefits of cooperative stag hunting. Additionally, we observed that this strategy was more efficient than the previous one. We thus revealed that the evolution of strategy which was individually adaptive (because more efficient) led to the transition to the optimal collective behaviour. Therefore, we showed that it was possible for individual selection alone to explain the optimization of group traits.

		In both of these studies, we thus reveal the critical role of coordination in the evolution of cooperation. In consequence, we demonstrated that it is indeed necessary to take the mechanics of coordination behaviours into account in order not to neglect crucial aspects of the evolutionary dynamics. Additionally, we also presented a general mechanism for the evolution of cooperation in the stag hunt. In our studies on the selection of equilibria, we can consider the boar to act as an evolutionary pathway in the stag hunt between hares and stags. This prey can be hunted alone but rewards more when it is hunted cooperatively, which is a realistic expectation for hunting in nature. As such, this prey can act as an initial bootstrap of coordination and cooperation. Then, we showed that when coordination was already evolved it was possible for individual selection to optimize the collective behaviour. In consequence, this could lead to the transition to the purely cooperative equilibrium: stag hunting.

	\subsection{Perspectives}


\section{Automatic Design of Collective Robots}

	\subsection{Contributions Synthesis}

		In the second part of this thesis our goal was to study the evolution of cooperation between robots in evolutionary robotics. More precisely, we were interested in the impact of genetic team composition on the evolution of efficient coordination strategies in multirobot systems. Multirobot systems have multiple advantages in comparison to using a single robot among which robustness, efficiency and the capacity to achieve tasks that a single robot could not. However, because they require the control of several agents, they are also harder to design. In this context, there exists multiple different architectures for multirobot systems and several ways to design the control of collective robots have been proposed. But while a popular and often efficient method has been to manually design the robots, there has been a strong interested in creating methods to automatically design them. Automatic design allows to build robots that could react better to environmental changes and efficiency. In this context, we focused on evolutionary robotics as an approach to the automatic design of distributed robots.

		An open issue when designing cooperative robots in evolutionary robotics is on the team composition. In particular, whether in terms of morphology or control, the robots that constitute a team can be homogeneous or heterogeneous. Here we only studied the influence of team composition on the control of agents. The classical approach in evolutionary robotics has been to use an homogeneous group of robots, where every individual comes from the same genotype. However, it is argued that heterogeneity could lead to more diverse behaviours between robots and thus lead to higher efficiency. We thus have been interested in contributing to this debate.

		In the first Chapter of this part, we compared in particular homogeneous and heterogeneous approaches on two criteria: evolvability and efficiency. We designed a collective foraging task inspired by the stag hunt where the individuals could forage two types of ressources: one that could be foraged alone and an other more rewarding that needed to be collected cooperatively. In this context, evolvability refers to capacity of a particular method to evolve cooperators while efficiency corresponds to the performance of the cooperative solution w.r.t. ressources collected. We compared a clonal approach (i.e. where individuals are homogeneous) to two aclonal ones (i.e. where the composition is heterogeneous): one where individuals are taken from the same population and the other where they come from two separately coevolved populations. We revealed that there exists a tradeoff between evolvability, which is best achieved with the clonal approach and efficiency, where coevolution evolved more efficient cooperative strategies. In particular we showed that division of labour would evolved with a coevolution approach. In order to go beyond this tradeoff and improve both methods on each criteria, we then added incremental evolution. The goal of incremental evolution is to decompose a complex task into several sub-tasks that are evolved separately as to ease the learning process. In our case, we pre-evolved our individual in a simpler cooperative task. We showed that while this offered no significant difference for the clonal approach, the evolvability of coevolution was greatly increased. In consequence we showed that an aclonal approach, coevolution, was the best method on both evolvability and efficiency. However, this increase in evolvability comes at the price of a pre-evolution step. We thus revealed a new tradeoff between: coevolution may outperform a clonal approach but at the cost of additional computations.

		In the next Chapter, we focused on the evolution of specialization under heterogeneous individuals. We took a simpler task than that of the previous Chapter where cooperation is easy to evolve but efficienty coordination strategies are favored. In this task, we showed that two coordination strategies could evolve: a turner strategy where both individuals are generalists and a leader/follower strategy where the two robots specialize. We thus wanted to study how division of labour could evolve under heterogeneous individuals at the level of the population. We thus wanted to evolve genotypic polymorphism, i.e. the coexistence of several different genotypes (encoding for diverse phenotypes) in a single population. We compared two selection schemes based on their capacity to achieve genotypic polymorphism: a \((\mu + \lambda)\) evolution strategy and fitness proportionate. We revealed that while specialists could easily evolve under elitist selection they could rarely be maintained throughout evolution. In comparison, fitness proportionate easily maintained genotypic polymorphism but was not efficient at evolving specialists. In order to understand the evolutionary dynamics at play, we then ran computational analyses based on the expected fitness of each strategy (turner, leader or follower) against every other strategy. We showed that generalists could invade the population by benefiting from the fact that specialists need to be paired with complementary specialists in order to be efficient. Additionally, we revealed that under small population size, some of the genetic diversity in the population could be lost during selection. From these results, we extracted two key properties for the evolution of genotypic polymorphism: stability of genotypic diversity and protection against the invasion of cheaters. While we could not achieve genotypic polymorphism in our study, we revealed that an algorithm validating these properties could achieve this goal.



	En concluding remarks ça peut être bien de dire que du coup Trianni2014b a raison, que l'ER ça peut être utile et qu'on espère qu'on a été convaincant etc...
\newpage
\section*{Abstract}

	Many different species behave in a cooperative fashion and cooperation is central to most of the major transitions in evolution. However, explaining its evolution is a major challenge in evolutionary biology. The evolution of altruistic actions in particular, where individuals pay a cost to bring benefits to others, has been widely studied. In comparison, mutually beneficial actions, which benefit every individual participating, have been relatively ignored. Yet, while this type of cooperation is stable once evolved, explaining its origin is a challenge, in particular when it requires the coordination of several individuals.

	In order to study the evolution of mutualistic cooperation, it is classical to use the evolutionary game theoretical model of the stag hunt. In this game, cooperating is more rewarding than acting in a solitary fashion but it is risky when rare. The issue is thus to study the emergence of the cooperative equilibrium. Here we claim that classical models in evolutionary biology make critical assumptions about the mechanics of behaviour that may impact the emergence of mutualistic cooperation. In consequence, we choose to address this issue with a framework that allows to take these mechanics into account: evolutionary robotics.

	The fields to which we contribute in this thesis are twofold. First we use evolutionary robotics to model the evolution of mutualistic cooperation. Taking inspiration from the game of the stag hunt, we design an experiment of collective hunting. We show that while the transition to cooperation is easy in a classical game theoretical model, this transition becomes impossible with our model in evolutionary robotics. We thus reveal how modeling the practical mechanics of behaviours impacts the emergence of mutualistic actions. Then we show how individual selection alone may optimize collective actions as the emergence of coordination allows the transition to the optimum. Additionally, we reveal that the nature of the coordination behaviours evolved impacts the probability for this transition to occur.

	In a second Part, we focus on the automatic design of controllers for distributed multirobot systems. More precisely, we study the influence of genetic team composition on the design of cooperative agents in evolutionary robotics. We first compare a clonal approach (i.e. homogeneous team) and two aclonal approaches (i.e. heterogeneous team) in a collective foraging task. We reveal the existence of a tradeoff between the capacity to evolve cooperation, best achieved with homogeneous robots, and the efficiency of the cooperative solutions, where the more efficient cooperators are evolved with a particular aclonal approach: cooperative coevolution. Then we focus on the issue of evolving specialisation among heterogeneous robots. We study how specialisation can evolve at the level of the population, i.e. genotypic polymorphism. We reveal the critical challenges raised by this issue and that for genotypic polymorphism to occur, it is necessary to protect against the invasion of generalists as well as maintain sufficient genetic diversity in the population.

	In conclusion, we show in this thesis how evolutionary robotics can contribute to a same problem (in our case the evolution of cooperation) in two very different directions: towards modeling in evolutionary biology or the automatic design of robots.
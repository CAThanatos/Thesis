\newpage
\section*{Abstract}

	Cooperation is central in the natural world and we can find examples of cooperative behaviours in many different species. However, explaining its evolution is a major subject in evolutionary biology. Indeed, the process of evolution is driven by the survival of the fittest individuals. Thus it would seems that the competition for spreading genetic material clashes with the emergence of behaviours that benefit others. As such, there has been a strong interest in explaining the stability of altruistic behaviours, where individuals pay a cost to bring a benefit to others, against the invasion of cheaters. In comparison, mutually beneficial actions, which benefit every individual participating, have been relatively ignored. Yet, the evolution of mutually beneficial cooperation raises another issue than that of stability. Indeed, because this type of cooperation benefits every individual, there is no selective advantage in cheating: mutualistic actions are stable once evolved. However, explaining their origin is a challenge. In particular, they may require the coordination of several individuals.

	In order to study the evolution of cooperation, a classical framework in evolutionary biology is to use evolutionary game theory. In the case of coordination a well-known model is that of the stag hunt. In it, two individuals can either hunt a hare in a solitary fashion or cooperate to capture a stag. Capturing a stag is more rewarding but also more risky. As such an interesting problem is to study the transition from the solitary equilibrium to the cooperative one. However, we claim that classical models in evolutionary biology make critical assumptions about the mechanics of behaviour (the proximate explanations) that may impact the emergence of mutualistic cooperation. In consequence, we choose to address this issue with a framework that models these mechanics: evolutionary robotics.

	While this thesis is focused on the main subject of the evolution of cooperation in evolutionary robotics, the fields to which our results contribute are twofold. First we use evolutionary robotics to model the evolution of mutualistic cooperation. Taking inspiration from the game of the stag hunt, we design an experiment of collective hunting in evolutionary robotics. We compare the results obtained with a classical model in game theory and show that there is a drastic difference when we model this problem with evolutionary robotics. More precisely, we show that while the transition to cooperation is easy in a game theoretical model, this transition is impossible in evolutionary robotics. We thus reveal that it is necessary to model the practical mechanics of coordination behaviours to fully address the emergence of mutualistic cooperation. Then we focus on the optimization of collective actions. More precisely, we study how can individual selection alone allow the transition from a suboptimal collective equilibrium to an optimal one. We show that, while this transition is impossible under simple environment conditions, the switch to the optimum can occur under more realistic assumptions. Additionally, we reveal that the nature of the coordination behaviours evolved impact the probability for this transition to occur.

	In a second part, we focus on the automatic design of controllers for distributed multirobot systems. More precisely, we are interested in the problem of designing cooperative robots with evolutionary robotics. We address the open issue of the role of genetic team composition in the evolution of cooperation. Namely we study the influence of homogeneous and heterogenous groups of robots on the evolution of cooperative agents. We first compare the results of a classical clonal approach (i.e. homogeneous team) and two aclonal approaches (i.e. heterogeneous approach) in a collective foraging task. We reveal the existence of a tradeoff between the capacity to evolve cooperation and the efficiency of the cooperative solutions. More precisely, the clonal approach evolves cooperative behaviours more easily but a particular aconal approach, cooperative coevolution, evolves more efficient cooperators. Then we focus on the issue of evolving specialization among heterogeneous robots. We want to study how specialization could evolve at the level of the population, which means that we want to achieve genotypic polymorphism. We reveal the critical challenges raised by this issue and that, to do so, an evolutionary algorithm needs to protect against the invasion of generalists as well as maintain sufficient genetic diversity in the population.

	In conclusion, in addition to the issues addressed in this thesis we show how can evolutionary robotics contribute to a same problem (in our case the evolution of cooperation) in two very different directions: towards the modeling of problems from evolutionary biology or toward the automatic design of robots.
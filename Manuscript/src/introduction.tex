\chapter{Introduction}

\minitoc[n] % minitoc without title

\section{The Evolution of Cooperation}

  \subsection{Cooperative Behaviours and their Importance in Evolutionary History}

    Among all of the social interactions present in nature, cooperation is the one that seems to be the most recurrent in any level of organisms' complexity. This is also a behaviour whose displays and pratical occurences are as diverse as useful for many different species. A most basic definition of cooperation is that it is a behaviour where an \emph{actor} (the individual who initiate the behaviour) will behave in such a way that is beneficial to a \emph{recipient} (the other individual with the behaviour)~\cite{West2006}. In particular, this behaviour will most often be costly for the actor. Obviously, in most acts of cooperation more than two individuals are directly concerned. And it can even actually be difficult to clearly define the actors and the recipients.

    But cooperation can be found at nearly any level of living complexity (TODO: PHRASING BOF). For example, even unicellular organisms such as bacteria are known to frequently act in a cooperative manner. Mostly based on secreting and using by-products, these organisms are capable of sharing and communication~\cite{West2006b}. A comparison is even often created between the way those bacteria exchange those by-product benefits to \emph{public goods} situations. These are are well known in the study of cooperative dilemma and in economy in particular. Moreover, it could be argued that the way genes interact together inside of a genome could fall under the umbrella of cooperation. Thus, even if in this case it could be debatable wether those are considered living organisms, it is clear that cooperation can be found at levels (TODO: bof le level) lower than could be imagined. Finally, multicellularity is a perfect example of high cooperation between numerous different organisms. Single prokaryotic organisms managed at some point to gather and incorporate as mitochondries what are now eukaryotic cells to form these complex multicellular organisms. This is considered no less as one of the major transitions in evolution~\cite{Szathmary1995}.

    If we focus our attention a little higher to organisms of a bigger size, an example which quickly comes in mind is eusociality~\ref{Wilson2005}. Of which the more obvious (but not only) representatives are insects like ants or termites. (TODO : def eusociality ?) These individuals have evolved one of the highest levels of sociality which implies that they constantly demonstrate very high levels of cooperation. Examples are numerous but among them individuals take cooperatively care of youngs which are not of their own (as they often are incapable of reproductive capabilities). Moreoever, they often display strong examples of division of labour, where individuals are able to divide between different to roles to achieve one or more complex tasks. In particular, in most eusocial organisms, this division of labour is permanent, leading to strong morphological differences between individuals of different behavioural groups (TODO : ref needed). Finally, the fact that most of the individuals in eusocial societies are denied reproduction is also an extreme form of cooperation.

    % TODO : mentionner les herds, flocks et schools ? En gros comportements collectifs
    % A lot of collective behaviours of larger animals can also be deemed cooperative. This is true even for animals who are not known to live in complex social structures. For example, the behaviours of fishes

    Again if we increase the scope and size of our study, we can find other examples of cooperations. Indeed among, the existence of strong sociality (what is called presociality) in a lot of animals comes with astonishing displays of cooperation (TODO : ref needed). These acts of cooperation may not be of less interest that those of a eusocial animals but they are closer to our understanding as human beings. Some mammal predators are capable of displaying amazing (TODO : bof) capacities for collective hunting, among them the obvious examples of wolves and lions (TODO : ref needed). But other more easily forgotten predators like hyenas (TD : RN) are known to rely on signaling and communication to create a complex coordination between up to a dozen of different members (TD : ouais bon vérifier le nombre quand même). The behaviours of those predators could easily be considered as complex strategy by a human observer. But even when hunting is not involved, cooperation is strong in these social animals. In a lot of species, individuals can act as lookout to watch for nearby predators. Most often, these members take the risk of attracting those same predators when they warn their companions, something which has been puzzling for a long time. Finally, even if the level of cooperation is not as strong as that of eusocial animals, we can find copious examples of cooperations inside these societies. Some adults take care of youngs which are not their own (TD:RN) which is known as allo-parenting or individuals partake in social bonding activities (like social grooming).

    What can be more stunning is to find cooperation between even members of different species. Again, examples are numerous. Again, sometimes this cooperation happens on a very small scale. In particular, the gut flora of some animals (including human beings) is constituted of bactera that help us process food. On a bigger scale we can think of cleaning fishes or the fact that some species of killer whales help human catch fishes, their benefit being they can prey on the birds attracted by the by-products of human fishing (TD:RN). Finally, domestication is another powerful example of interspecies cooperation. Interestingly, domestication can be found in other animals than human (e.g. ants and aphids TD:RN).

    In conclusion, in nearly every different levels of complexity in the living world, we can find displays of cooperation. More importantly, cooperation is not only present but also seems to be one of the factors for every major transitions in evolution~\ref{Szathmary1995}. Yet its evolution can be hard to explain at first glance. Indeed, according to darwinian evolution, a behaviour should be selected only if it benefits the individual (TD:RN ? mon poto Darwin ?). This obviously seems to contradict the very definition of cooperation. This is thus a problem which has (and still is) of major interest for evolutionary biology.


    % TODO : Darwing parlait déjà du problème de la coopération je crois non ?
    % TODO : caser les humains dans le tas encore ?

  \subsection{Altruism and Indirect Fitness Benefits}

  \subsection{Mutualism and Direct Fitness Benefits}
    \begin{itemize}
      \item{At the end of this subsection, readers should be convinced that intraspecies mutualism is (1) important and (2) not really studied}
    \end{itemize}


\section{Coordination in Cooperative Behaviours}
  
  \subsection{Collective Acts of Coordination}
    \begin{itemize}
      \item{More of a review of coordination. The idea would be to show the strong link between coordination and mutualism in nature.}
    \end{itemize}
  
  \subsection{The Dilemma of Coordination}
    \begin{itemize}
      \item{Chicken \& Egg problem + explnation of the Stag Hunt}
    \end{itemize}


\section{Evolving Coordination in Evolutionary Robotics}
  
  \subsection{Evolutionary Robotics}
    \begin{itemize}
      \item{Basic explanation of ER}
    \end{itemize}
  
  \subsection{Modelling the Evolution of Cooperation}

  \subsection{Designing Cooperative Robots}
    \begin{itemize}
      \item{Both last subsections are brief introductions to each part of the thesis}
    \end{itemize}

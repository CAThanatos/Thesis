\section{Introduction}

  In this chapter, our goal is to model the proximate mechanisms in the evolution of coordination and study their effects on the ultimate emergence of cooperation. As in most fields of science, modeling is a standard approach to study problems that cannot be understood by observing the physical world. Thus models are now commonly accepted in the realm of evolutionary biology, even if this may not have been the case for a longer period of time than in other scientific fields (e.g. physics)~\parencite{Shou2015}. 

  As said in the Introduction of this manuscript, we choose evolutionary robotics as our modeling technique to this problem. However, we deliberately only briefly justified this choice hitherto. This manuscript is composed of two main parts which, while related, represent different scientific questions in very different fields. As such, we consider that it would be more convenient for reading that any justification about the modeling choice would be placed here. Thus, the next Subsections will be devoted to that task. We will first focus on the empirical works that have been interested in studying evolution and the evolution of cooperation in general. Then we will talk about the different modeling approaches used in the field of evolutionary biology. To that end, we will distinguish between the more classical methods of modeling in evolution and the computational methods that arose with the availability of computers. This will finally give us the opportunity to motivate the use of evolutionary robotics in light of the set of models available.


  \subsection{Why Model the Evolution of Cooperation}

    In this chapter, our approach is one of modeling the evolution of cooperation. In this present Subsection however, we are interested in briefly reviewing the more empirical study of evolution. In this way, our goal is thus to motivate the common need to use models in understanding evolutionary phenomenons.

    Everything that we contemplate now is the consequence of thousands of millions years of evolution; the earliest appearance of life is dated to $4000$ millions years ago and a little more than $2000$ millions for the eukaryotes. The time scales involved in the evolutionary process are such that we cannot directly observe evolution but rather try to understand how it could have taken place. This means that we are mostly left with evidences from the past. In consequences, a large part of our empirical understanding comes from paleontologists. A copious amount of work in particular has been dedicated to studying the social life of our hominin ancestors. For example, paleontological were used to know the diet of early hominins. More importantly, those records reveal a strong correlation in our ancestors between the switch to a more energy-rich diet (e.g. meat) and the increase in the brain size of these ancestors~\parencite{Aiello1995, Wrangham1999}.

    This would suggest a coevolution between the transition to a rich diet and the appearance of social behaviours (which require higher brain functions~\parencite{Dunbar2007, Isler2012}). Theories on how exactly this transition shaped cooperation are diverse~\parencite{Pontzer2012}. Some think that this new diet could allow to adopt more diverse and more efficient hunting and scavenging techniques. Indeed, a richer diet would be necessary to develop the locomotive and cognitive skills necessary for both of these activities~\parencite{Aiello1995, Bramble2004}. Others argue that a shift in climatic conditions led the early hominins to rely more on underground storage organs. While this type of food is rich and available even under arid climate, its harvesting is difficult. This drove the selection for the capacity to share food among individuals, thus leading to cooperation~\parencite{OConnell2002}. It is also believed this led to cooking food which created a physical location where individuals would meet and thus create social dynamics~\parencite{Wrangham1999, Wrangham2009}. Therefore, those paleontologic evidence can be used to come up with hypothesis on how the selective pressures resulting from ecological conditions could shape the evolution of social behaviours.

    However, most of the empirical work on cooperation is achieved by observing animal behaviours (i.e. ethology). In particular, it is believed that we could come with essential explanations of the evolution of sociality in humans by studying that of non-humans. For example, persuasive links have been made between the evolution of cooperation in humans and in social carnivores~\parencite{Schaller1969, Smith2012a}, another strong argument in favor of a correlation between the transition to a rich diet comprised of meat and the evolution of social behaviours.

    As previously stated in the Introduction, most research on the evolution of cooperation have been focused altruism. And as such, a large body of work has been dedicated to finding evidence of kin selection in animals~\parencite{Bourke2014} and in eusocial insects in particular. Only $2\%$ of insect species are eusocial but they represent most of the insect biomass~\parencite{Wilson2008}, the most obvious of representatives belonging to hymenoptera order (e.g. ants, bees or wasps). In particular, one particularily difficult behaviour to explain is the presence of reproductive division of labour, where only a certain "caste" of individuals will reproduce~\parencite{Wilson1990}. Empirical works on eusocial insects have shown that colonies are composed of strongly related organisms. In particular, all membres of the colony are offsprings of a single individual (i.e. the queen). This confirmed kin selection as a mechanism to explain the evolution of eusociality~\parencite{Queller1998}. Other empirical works have also investigated the action of kin selection in vertebrates. In particular, it has been shown that in most cooperative species of birds and mammals, social groups are small and composed of dominant breeders and their relatives~\parencite{Dugatkin1997, Clutton-Brock2002}. In that case, kin selection was brought upon to explain differences in breeding (i.e. reproductive skew) between individuals of the same group and the rearing of youngs by individuals which are not their parents~\parencite{Bourke2011}. However, whereas kin selection is necessary to explain the evolution of such altruism in eusociality, the importance of kin selection among vertebrates is much more debated~\parencite{Griffin2003, Clutton-Brock2002}.

    Field studies on verterbrates have also focused on the role of direct fitness benefits in cooperation. As we previously mentioned, most of cooperative actions are likely to result from a combination of both indirect and direct benefits~\parencite{Clutton-Brock2009}. Thus, it may be hard to distinguish with certainty occurences of cooperative actions that produce direct benefits. However, the presence of intraspecific cooperation between unrelated individuals may only be explained by direct benefits to fitness. For example, spotted hyenas have been shown to cooperate with kin and non-kin alike to achieve collective hunting and defense against other predators~\parencite{Drea2009a, Smith2010, Smith2012a}. Thanks to field work on cooperative vertebrates, various hypothesis on how direct benefits are obtained through social interactions have been proposed~\parencite{Clutton-Brock2002}. For example, the existence of cooperative breeding can be explained by group augmentation. This means that individuals have a direct benefit in increasing group size: better protection against predators, more efficient gathering and defense of food and rearing of youngs~\parencite{Packer2001}. More generally, multiple cooperative actions can be maintained through mutual benefits. Primates are known to engage in mutual grooming and birds can build communal nests which directly benefit every participants. Obviously collective hunting can profit to every individuals by increasing the product of the hunt and decreasing the risk (both in term of being wounded by the prey and not being able to catch it)~\parencite{Scheel1991}. Cooperation has also been shown to be enforced through coercion, where non-cooperative individuals could suffer punishment or eviction from the group. Reciprocity~\parencite{Trivers1971}, where individuals will cooperate with partners with which they previously interacted has often been proposed to explain cooperation between non-kin. However at the moment, there is no sufficient evidence that this type of behaviour occurs in other vertebrates than humans~\parencite{Hammerstein2003, Clutton-Brock2009, Andre2014}.

    % Je mentionne peu les primates ici. Ca serait bien d'en parler très rapidement (après tout on apprend pas beaucoup de choses vraiment différentes (voir même moins qu'avec les hyènes d'après Drea2009a))

    It is also worth mentioning briefly studies on cooperation in humans. Economists have been interested on human altruism and have produced experiments to display the existence of altruistic cooperation between non-kind~\parencite{Fehr2002, Fehr2003a}. Because kin selection cannot be involved here, altruism in human societies is best explained by a pattern of rewards, reputation and punishment. In particular, they showed the importance of strong reciprocity between humans. While altruistic reciprocator will reward and punish individuals if they have a long-term interest~\parencite{Trivers1971}, strong reciprocators will do so in any case (i.e. in a stricly altruistic way). Fehr and Fischbacher~\parencite{Fehr2003a} argue that the evolution of human altruism cannot be only explained by gene-based evolutionary theory alone and advocate for gene-culture coevolution. Additionnaly, some have also been interested in the coordination in mutualistic actions between human beings. For example, Michael Alvard has studied traditional societies of whale hunters in Lamalera, Indonesia~\parencite{Alvard1999, Alvard2003}. In particular, he has shown that there is no relation between strong kinship and increase in cooperation in these societies. Alvard thus argues for mutualistic explanations for the evolution of coordinated actions. Moreoever, other studies have also been focused on the mechanisms with which humans coordinate to solve cooperation problems. For example, Bullinger and colleagues~\parencite{Bullinger2011, Duguid2014} have studied how exactly chimpanzees and human childs coordinate in a stag hunt type of game. They have mainly focused on the role of communication and leadership strategies between individuals.

    Lastly, we need to mention the subset of empirical work where cooperation is directly observed while it takes place. This can only done in organism where evolution is a fast process: microorganisms~\parencite{Elena2003}. For example, Richard Lenski conducted his "Long-Term Evolution Experiment" on Escherichia coli~\parencite{Fox2015}. This experiment began in 1988, when Lenski set up $12$ populations of the same genotype of E. coli and observed their evolution since then. But more generally, bacteria regularly act in ways that can be caracterized as cooperative~\parencite{West2006}. We previously talked about public goods games in \emph{P. aeruginosas}. But some cells of the slime mould Dictyostelium discoideum have been observed to sacrifice themselves so that other cells can become spores and thus spread their genes~\parencite{Strassmann2000}. Microorganisms have thus often empirically validated kin selection and kin discrimination~\parencite{West2006}. In particular, while green beard mechanisms are thought to be rare, \emph{D. discoideum} are an exemple of this mechanisms. In this case, one gene encodes for individuals to stick to each other to cooperatively form fruiting bodies~\parencite{Queller2003}. But evidence of direct benefits between microorganisms have also been revealed. Mutualism (i.e. interspecific cooperation) is frequent between microorganisms where it can provide direct by-product benefits. Evidence of punishing behaviours have also been shown in mutualism between plants and bacteria, like between leguminous plants and the rhizobial bacteria~\parencite{Kiers2003}. In this mutualism, the bacteria costly fix $N_{2}$ which is supplied to the plant. The plant enforces cooperation by decreasing the supply of $O_{2}$ given to the bacteria should it defect. 

    %To conclude, it also interesting to note that there are pratical medical applications for studying the evolution of cooperation in microorganisms as there exists a correlation between cooperation and the virulence of bacteria strains~\parencite{Foster2005}.

    In conclusion, empirical studies on the evolution of cooperation exist and are widespread. However, they rarely inform on the mechanistic constraints involved in the evolution of coordination: proximate explanations are difficult to understand when you can only observe the ultimate results of such a long process. In comparison, in the particular case of microorganisms, microbiologists are concerned with proximate explanations of cooperation~\parencite{West2006}. However, they do not unfortunately offer the wide variety of coordination behaviours that what can be observed in bigger organisms. Moreover, cooperation in microorganisms is based in majority on kin selection mechanisms or interspecific mutualism.

    % Parler de ce que dit Smith sur protection against predators ?
    % Man the hunted

    % Phylogénétique ?

  \subsection{Classical Modeling in Biology}

    As previously said, it is nowadays classical in evolutionary biology to model the evolution of cooperation (among other evolutionary traits). As previously stated, it also necessary to do so to fully grasp the mechanisms at play.  Models are numerous and a lot of different frameworks have been proposed to classify cooperative actions~\parencite{Dugatkin2002, Sachs2004, Lehmann2006}. To that end, mathematical models dominate the field~\parencite{Servedio2014}. 

    In particular, \emph{population genetics} form a large part of the litterature in evolutionary biology. The inception of this field was mostly the result of Ronald Fisher, J.B.S. Haldane and Sewal Wright. Fisher was the first to link mendelian inheritance (i.e. the inheritance of biological traits) with mathematical models of natural selection in his book \textit{The Genetical Theory of Natural Selection}~\parencite{Fisher1930}. Population genetics is concerned with studying the changes in alleles' frequencies at a particular locus in the genotype. In particular, the focus is put on population wide variations of evolutionary traits within one or a few loci. As such, there has been a strong emphasis on studying genetic mechanisms like dominance, epistasis or genetic drift. A notable branch of this field, \emph{quantitative genetics} has in comparison been more focused on the phenotypical aspects of evolution. More precisely, quantitative genetics deal with continuously varying phenotypical trait. As such, it abstracts from the genetic details of evolution. However, population and quantitative genetics both tend to abstract from the role of ecological features, which we are interested in here.

    % -> Parler de kin selection ici
    % -> Good for ultimate explanations
    % -> Introduire modèles de Moran et Wright-Fisher ? (et probablement équation fondamentale de Fisher. En gros montrer à quel point c'est mathématique ET fondamentale) 

    % Reprendre la partie sur population genetics

    % Dire bien que pop gen ça nous intéresse pas vraiment vu qu'on s'intéresse au phénotype

    In comparison to population genetics, \emph{evolutionary game theory} (EGT) puts a strong emphasis on ecology. Game theory was originally conceived by the mathematician John von Neumann as a way to determine the optimal strategies in a contest between several (usually two) "players". Given a payoff matrix, each player can expect a certain payoff depending on her strategy and those of the other players. In this framework, players are expected to be rational and follow this optimal strategies. One of the most important concepts of game theory is the \emph{Nash equilibrium}~\parencite{Nash1950}. In game theory, an equilibrium represents a set of the strategies played be each player. Under a Nash equilibrium, no player benefits in deviating from her strategy if the other players keep their strategy. This framework was first adapted to darwinian evolution by John Maynard Smith and George Price under the name of evolutionary game theory~\parencite{MaynardSmith1973}. The novel idea is that the players, rather than choosing a strategy based on a rational decision, simply play a strategy based on their phenotypes. Therefore, an individual's strategy is now inherited and we study the evolutionary perspective of the different strategies. To that goal, the payoff matrix of an evolutionary game corresponds to the fitness value of each strategy. In order to study the evolutionary dynamics of an evolutionary game, we study a population of individuals all playing the same strategy. We then consider the appearance of a rare mutant who plays a different strategy and study the fate of these two strategies. If the strategy of the mutant has a higher fitness than that of the initial population (called the \emph{resident strategy}), then it may invade the population and replace the resident strategy. Otherwise, the mutant strategy will be selected against and disappear. If this resident strategy is stable against any mutant strategy, we say that this strategy is evolutionarily stable (ESS)~\parencite{MaynardSmith1973}. Interestingly, all ESS are Nash equilibria (but the opposite may not hold).

    One of the main feature of evolutionary game theory in comparison to population genetics is that it takes into account the influence of one individual's behaviour on the fitness of others. More precisely, the fitness of an individual will depend on the proportions and behaviours of other individuals in the population, which is known as \emph{frequency-dependent selection}. As such, EGT is convenient to account for the ecological features of a particular evolutionary phenomenon~\parencite{Hammerstein1994}. Because it represents conflictual and cooperative interactions, EGT has from been of great interest in studying the evolution of social behaviours since its inception~\parencite{Bshary2015}. One famous example is that of reciprocity in the \emph{Iterated Prisoner's Dilemma}~\parencite{Axelrod1984}. The prisoner's dilemma was already a famous game theoretic model but Axelrod \& Hamilton proposed that individuals play an iterated version of this game (of which you can find Axelrod \& Hamilton's payoff matrix in Table~\ref{table:payoffIPD}). To put it more simply, when individuals have engaged in a game interaction, there is a probability that they will meet again in a later interaction. They showed that, under those circumstances the evolutionary stable strategy is one called \emph{Tit for Tat} (TFT). Under this strategy an individual always cooperate when meeting an opponent for the first time. It then always copy the opponent's last move which means that (1) it retaliates and (2) it does not hold grudges. This strategy was thus presented as a theoretical example of reciprocity~\parencite{Trivers1971}. More generally, a large body of work has again been focused on the evolution of altruism when faced with the appearance of free-riders (defectors) in the prisoner's dilemma~\parencite{Requejo2013a}.


    \begin{table}[ht]
    \centering
      \caption{\textbf{Payoff matrix of the prisoner's dilemma.}}
      \begin{tabular}{l|c|c}
        & Cooperation & Defection \\
        \hline
        Cooperation & 3,3 & 0,5 \\
        \hline
        Defection & 5,0 & 1,1 \\
        \hline
      \end{tabular}
      \begin{flushleft} The strategy of player A (resp. player B) is symbolized by each row (resp. column). The payoff of player A (resp. player B) is shown on the left (resp. right). This is the payoff matrix which was used by Axelrod \& Hamilton in their work on Tit for Tat in the prisoner's dilemma~\parencite{Axelrod1984}. 
      \end{flushleft}
    \label{table:payoffIPD}
    \end{table}


    In the case of coordination which we are interested in here, the prisoner's dilemma is not appropriate to model the evolutionary dynamics of cooperation~\parencite{Alvard2002, Skyrms2004}. This is why there has interest in coordination games and in particular the stag hunt~\parencite{Skyrms2004, Requejo2013a}. We will not go in detail here in describing this game as it was already done in the Introduction. What is important to know is that the major difference with the prisoner's dilemma is the presence of a second ESS which is the cooperative equilibrium. Thus the emphasis of this game is not on the risk of invasion of a population of cooperators by free-riders (as cooperation is stable once evolved) but on the transition from the solitary equilibrium to the cooperative one. In his original book, Skyrms mainly studied the influence of location, signaling and partner choice~\parencite{Skyrms2004}. While the literature on coordination games is much smaller than that on the prisoner's dilemma, multiple works have ben interested on this subject~\parencite{Santos2006, Pacheco2009, Iyer2016}.

    % -> Mentionner les différents types de jeux ? (coordination games, PD etc... (ya un papier là-dessus))

    But in classical EGT, strategies are discrete and constitute a finite list. This is most often called matrix games. In comparison, most traits in evolutionary ecology take values in a continuous domain. We can for example think about the size, flowering rate or the investment and allocation of resources~\parencite{McGill2007}. In order to study the evolutionary dynamics of those traits, a continuous version of EGT rose as \emph{adaptive dynamics} (or sometimes simply continuous-trait game theory). This modeling technique can be seen as a way to combine population genetics, by studying the rate of change of a population's strategy, and EGT, by using the concept of frequency-dependent selection~\parencite{Geritz1998, McGill2007}. More precisely, adaptive dynamics extends on the main notion of EGT: the evolutionary stable strategies. In particular, the concept of ESS as it exists in EGT lacks any knowledge about the convergence of a given strategy. To put it more simply, we know that a strategy which is ESS will not be invaded by any mutant strategy once it has spread in the population. Yet, we do not know if this strategy will actually become established. This later dynamic is defined by \emph{convergence stability}, a notion which defines the fact that a strategy, thanks to multiple small evolutionary steps, will be able to emerge. Both concepts of evolutionary stability and convergence stability do not always come together~\parencite{Eshel1981, Eshel1983}. Behind convergence stability is the idea that the shape of the fitness landscape changes as the population's strategy is changing. From this it stems that it may be impossible to evolve an ESS even when it may represents a fitness maximum.

    Several key evolutionary concepts that could not be modeled by classical EGT have been introduced through the framework of adaptive dynamics. For example, one such concept is that of "branching points". These occurs when a strategy is convergence stable but not evolutionarily stable~\parencite{Geritz1998}. Namely this strategy acts as evolutionary attractor from afar but, because the fitness landscape changes as the resident strategy changes, this strategy may be a fitness minimum (and thus not ESS). These are called branching points because two different evolving population may coexist and evolve separately. Branching points have thus been used to model the evolution of speciation~\parencite{Geritz2004}. Coevolution has also been widely studied in this field, thanks to the modeling of frequency-dependence. In particular, some have been interested in the competitive coevolution of predators and preys and how branching points influence the apperance of niches in these systems~\parencite{Bowers2003}. Cooperative coevolution (i.e. interspectific mutualism) has also been studied. More precisely, one of the main aspects here is the incentive for cheaters to profit from the benefits of mutualistic interactions. Thus some have been interested in studying the stability of mutualism against cheating~\parencite{Ferriere2002, McGill2005}. Finally, it is without surprise that the evolution of altruism represents a fair share of research in adaptive dynamics. Most notably, the ecological modeling in adaptive dynamics (as in game theory in general) has allowed to study how dispersal could be a prominent mechanism to decrease kin competition, thus effectively enabling the evolution of altruism~\parencite{LeGalliard2003, LeGalliard2005}.

    % Foutre une mini conclusion ici ?
    To conclude, there is thus a wide range of mathematical models that deal with evolutionary questions.

    % Main criticism But the thing is beyond an invasion threshold 
    

  % Parler plutôt de Computational Biology ? J'aime bien l'idée de parler de techniques de modélisations particulières qui s'inscrivent finalement dans cette opposition plus générale de Classical VS Computational
  \subsection{Individual-Based Modeling and Evolutionary Robotics}

    The mathematical models presented before are oftentimes named "classical models"~\parencite{DeAngelis2005, Adami2014} and this how we will refer to them in this manuscript. It is important to note that this term is not used in a derogatory fashion to distinguish between. Rather, this is a way to discriminate between the purely analytical models which have been classical in evolutionary biology and a range of models which were born thanks to an easier access to computational power. This allowed to approach biological questions in a different way that, some would argue, enables to go beyond what is possible with purely mathematical modeling~\parencite{Adami2012}. However, the line between classical and computational models can sometimes be not so easy to draw and there is a real scientific interest in trying to get the best of both worlds~\parencite{Wilson1998}.

    \subsubsection{Individual-based models.} A type of computational methods which garnered a strong interest in behavioral ecology are \emph{individual-based models}\footnote{The term \emph{agent-based model} can often be found in the litterature in lieu of individual-based model. Both names refer to the same technique and can be used interchangeably. While individual-based model (IBM) are found more often in biological applications~\parencite{Grimm2005}, no real consensus exists on which term to use. We choose to use the latter throughout this manuscript.}~\parencite{Huston1988}. As stated in the name, an IBM focus on modeling the individual. This is very different from most classical models in ecology where the emphasis is mainly put on population dynamics~\parencite{Grimm2005}. This does not mean that research in IBMs are not interested in studying population dynamics but rather that these dynamics are studied as a consequence from the interactions between individuals. To put it more simply the main focus of an IBM is to study the collective dynamics emerging from individual-level interactions (whether with other individuals or the environment). And more importantly, the particularity of an IBM is that these individuals, which are the building blocks of the system, are adaptive: collective properties arise from these (sometimes simple) adaptive behaviours.

    IBMs are mostly used in studying biological phenomenons for which individual variations, and the assumptions that stem from them, are critical. DeAngelis \& Mooij~\parencite{DeAngelis2005} defined five axes along which IBMs are used to model mechanistic details in the variations between individuals:

    \begin{description}
        \item[Spatial variability] {While classical models sometimes take into account spatial organization, IBMs allow to model local heterogeneity between individuals.}
        \item[Life cycle] {The variability of ontogentic history can be model with finer detail by using IBMs than with classical models.}
        \item[Phenotypical variation and plasticity] {DeAngelis \& Mooij showcase the influence of individual experience on the behaviour. In particular, IBMs can be useful to model the interactions of multiple (more than three) different behaviours than classical (game theoretical) models.}
        \item[Learning] {Learning is obviously a consequence of lifetime interactions which is dependent on individual variations.}
        \item[Genetics and evolution] {The computational power of IBMs can help study complex evolutionary genetics.}
    \end{description}

    Thus IBMs have been widely used in behavioural ecology for diverse applications~\parencite{DeAngelis2005}. For example, spatial variability has been of great interest for the study of groups patterns. Most notably, models on the formation of groups of animals, whether swarms of insects, flocks of birds, herds of mammals or schools of fishes~\parencite{Huth1992, Gueron1996, Couzin2002} rely heavely on the framework put forth by IBMs. In particular, aggregation behaviours were found to easily emerge from simple local (individual) sets of rules, leading to complex collective behaviours. This is a perfect illustration of one of the most basic principle of IBMs. But more generally, IBMs have been used to model ecological phenomenon as diverse as the optimum gap between trees in forests models~\parencite{Botkin1972}, movement patterns in prey-predators interactions~\parencite{Smith1991} or differences in foraging between solitary birds and large flocks~\parencite{Toquenaga1995}.

    Individual-level variations in IBMs has also lead to multiple works in evolutionary ecology. By putting the emphasis on the invidiual, it is thus possible to clearly study how individual adaptations can lead to the fixation of evolutionary traits at the level of the population. In particular, IBMs can be used to model proximate mechanisms and study their ultimate evolutionary consequences. From this it stems that the evolution of cooperation, as one of the trademark problems in evolutionary ecology, has also been studied in IBMs. Our goal here here is to briefly cover a few diverse examples of IBMs filling that role. For example, Olson and colleagues have been interested in the evolution of herds. More precisely, while collective aggregation benefits the individuals in the group, it is also costly for them (i.e. sharing ressources and increasing the risk of predation). Thus there is an evolutionary question on the evolution of such collective behaviours. They confirmed that the formation of herds could be explained by Hamilton's theory of the selfish herd (i.e. an aggregation is created because every individuals try to put the other individuals closer to the predator)~\parencite{Hamilton1971, Olson2013a}. They also showed that a mechanism of predator's confusion could explain this behaviour~\parencite{Olson2013} and that it could lead to a coevolution for both predator and prey of morphology and behaviour~\parencite{Olson2016}. Finally they also revealed that group vigilance could explain the formation of gregarious foraging behaviours without any kinship relations~\parencite{Haley2014, Olson2014a}. Other have been interested in symbiogenesis, which refers to the creation of a new species by the symbiosis of previously independent species. Watson et al. showed with an IBM that such process could occur without any relatedness between the individuals~\parencite{Watson1992}. In comparison, Wilder \& Stanley have been interested in modeling the evolution of altruism, both with an IBM and a classical analytical model, thanks to ecological niches~\parencite{Wilder2015}.

    % Citer Tarapore2010 ?

    Finally, it is important to note that interplays between IBM and EGT are numerous. Indeed, both methods are focused on phenotypical evolution and rely heavily on the ecological features of the system. This led to a great number of research bringing together these two fields "with ease". In fact, even the most basic of IBMs may allow to lose some of the assumptions made by  classical models in EGT~\parencite{Adami2014}. In particular, it is now classical to take spatial interactions into account (e.g. how individuals are located on a graph) when studying the evolution of (mostly altruistic) cooperation~\parencite{Hauert2004}. But IBMs also give the possibility to more accurately predict the effects of a finite population size (while most models in EGT use infinite population sizes~\parencite{Hauert2009}). Additionnaly, it is also possible to more easily model stochastic or conditional strategies. These ecological features can be modeled with classical game theoretical models but at the cost of increased mathematical complexity~\parencite{Hauert2009}. Moreoever, modeling these features with classical models often force to consider simplified situations compared to what can be achieved with IBMs (e.g. having no more than three different strategies). Lastly, individual-level modeling implies that most often, minimal assumptions are made on the effects of mutations on behaviour. In conclusion, IBMs represent a strong addition to EGT when we are interested in the modeling of proximate mechanisms.

    % Il va falloir caser une réflexion plus générale sur les modèles Alife tels qu'AVIDA et Aevol puisque ce qu'on fait c'est de l'Alife, pas de la modélisation de véritables organismes.
    % Pour ça les modèles sont très divers aussi, par exemple AVIDA et Aevol (sans trop s'appesentir ni sur l'un ni sur l'autre) -> Dire que ya un historique Artificial Life là-dessus : Ou plutôt commencer sur le fait que tu PEUX avoir un côté très artificial life (mais est-ce que c'est vraiment du modeling à ce moment-là ? Oui, débatable)
    %     -> En gros la frontière est compliquée
    % Il y a quand même une différence entre de l'IBM pour modéliser des vrais animaux et de l'IBM Artificial Life (donc en gros un peu ce que dit Webb). Va falloir l'expliquer ça à un moment.
    % IBM are diverse ? (genre AVIDA c'est IBM, AEVOL aussi)
    % -> Sometimes closer to reality -> aevol, sometimes further -> AVIDA
    % Si besoin, dans Floreano2010 ya un peu de biblio sur AVIDA (ça sera bien suffisant !)
    % Digital genetics: unravelling the genetic basis of evolution (à garder de côté pour Avida et ALIFE)
    % The ecology of action selection: insights from artificial life Anil K. Seth Pareil (pour ALIFE, pas AVIDA)


    \subsubsection{Evolutionary robotics.} We will now finally talk about \emph{evolutionary robotics}~\parencite{Nolfi2000, Doncieux2015a} (ER). We will not get here into the technical details of this framework as this was already covered in the Introduction of the manuscript. What we want to talk about here are the reasons for which one could use evolutionary robotics above any other modeling techniques presented before. To summarize quickly what we said in the Introduction, behind evolutionary robotics is the idea of using the Darwinian principles of selection and variation to evolve a robot's controler and/or morphology~\parencite{Floreano2010, Doncieux2014}. At its core, evolutionary robotics is a design technique which stems from the need to bring an holistic approach to the conception of a robot. Because creating a fully embodied agent is complex, the key principle is to let evolution do the work. So why should evolutionary biologists be interested in using ER ?

    The reasons are in fact very similar to those of IBMs. Individual-based modeling and evolutionary robotics have actually very comparable features that makes them appealing as a framework. Namely, they both model evolution at the level of the individual. This means that there is a focus on the adaptive behaviour of an individual given a particular ecological context. Thus, there is an abstraction from underlying genetic mechanisms of evolution. On the contrary, the emphasis is put on the evolution of phenotypical traits. As in IBMs, ER can be used to model spatial organization and local interactions, life cycle dynamics and any kind of phenotypical plasticity~\parencite{Mitri2012}. ER is thus also appropriate in the study of behavioural mechanisms and their impact on evolution. Therefore, the threshold between ER and IBMs can sometimes be blurry. It can even be argued that some studies conducted in ER could have led to similar conclusions with a "simpler" IBM~\parencite{Mitri2012}.

    But that does not mean that there is no profound differences that may motivate the use of ER over IBMs. The main difference between these two frameworks is summarized by the following assertion: ER models are IBMs where the individual is an embodied robot~\parencite{Mitri2012}. By definition, robots have a physical body. In consequence, this creates an additional level of interactions with the environment. Sensory feed ack is also part of a robot's design, which means that there often is imperfect information about the environment. In comparison, IBMs will often (but not always) provide a global and perfect description of the world. Finally, modeling in ER implies that the environment (whether simulated or not) exists in a bounded space, which may not be the case in IBMs. All of this can have lasting consequences on the dynamics of a system. When those physical properties are expected to be of importance for the phenomenon at play, it may be advantageous to use ER rather than IBMs. Mitri and colleagues~\parencite{Mitri2009} provide an interesting example of a situation where physicial embodiement led to surprising results. While their study was focused on the evolution of communication between simulated robots for foraging, they found that the aggregation of robots on a foraging site provided additional information that did not require the use of communication. More generally, as we previously said, the line between IBMs and ER may be blurry. Oftentimes, the differences between both framework mainly rests upon terminology and historical basis. Namely, these two techniques come from (at least originally) different communities. As we previously explained, ER is deeply rooted in the field of robotics design. In comparison, a large part of IBM is interested in the design multi-agent systems and their applications. While ER and IBM can be very similar, these historical divergences tend to have a lasting effect. In comparison, we could simply consider ER to be a particular instance of IBM.

    There has been an extensive, though recent, effort in using ER as a modeling tool for social behaviours as well as evolution~\parencite{Mitri2012, Trianni2014b, Eiben2014, Doncieux2015a}. However there still is a lack of communication between communities which implies that ER research sometimes fail to reach those who could be interested by these findings in the evolutionary biology community. This also means that some ER researcher may sometimes focus on questions that are of no particular relevance for evolutionary biologists~\parencite{Trianni2014b, Doncieux2015a}. Here we want to present a quick overview of those more significant works in ER that have been interested in modeling the evolution of social behaviours. First, the evolution of communication has been a major subject of interest in ER. This framework is (like IBM) designed for modeling at the individual-level, which is adamant in understanding the population-level evolution of communication. And, as we talked about previously, the additional embodiement of individuals in space can have some unexpected effects in communication behaviours~\parencite{Mitri2009}. For example, Floreano \& colleagues~\parencite{Floreano2007} showed how the evolution of communication could vary depending on the relatedness inside a group of foraging robots. Communication could easily evolve when there was strong relatedness (i.e. robots were clones of each other). In comparison, when individuals were unrelated, deceptive strategies would also evolve. Similarly, Mitri et al.~\parencite{Mitri2011} showed a strong correlation between signal reliability and relatedness between individuals. Wischmann, Floreano and Keller~\parencite{Wischmann2012} conducted a study where they revealed that purely historical contingencies could lead to diverging communication strategies in independant evolutionary runs. They showed in particular that a more or less complex signaling strategy could evolve based on these contingencies. In particular, the more complex strategy would not ensure higher performance unless in a competitive setting between these two strategies. Finally, in another study we briefly talked about before, Mitri and colleagues~\parencite{Mitri2009} have focused on the evolution of both communication and suppression of signaling in a competitive environment. Then, also on the subject of communication, Solomon et al.~\parencite{Solomon2012} produced a very compelling study on the evolution of signaling strategies in cooperative robots. They took inspiration from the behaviours of spotted hyenas~\parencite{Smith2010} when they compete against lions for stealing a prey. In particular, they compared the performance of two different signaling strategies: (1) one where all individuals can signal to others and (2) another one where only a particular hyena, the flag-bearer, can signal. The latter revealed to achieve higher coordination between individuals. It is interesting to note that this study is one of the rare models in ER where individuals are all unrelated (as we will talk about later). Some have been interested in the evolution of swarming behaviours. We talked about this previously during our presentation of IBMs but Olson \& colleagues have studied the evolutionary mechanisms behind the emergence of herding behaviours~\parencite{Olson2013, Olson2013a, Haley2014}. While they categorize their work as IBM, it could argued that their model belongs to the field of ER. This is another evidence of the interplay between both communities. Quite differently, Elfwing \& Doya~\parencite{Elfwing2014a} have been interested in the evolution of polymorphism, where individuals with different phenotypes coexist in the same population at the same time. More precisely, they simulated an environment filled with energy sources that an individual needs to collect to ensure its survival. Moreover, the individuals had to physically exchange genotypes with other individuals in order to simulate mating and produce offsprings. The individuals could choose between two different strategies, focusing (1) on the energy sources or (2) on tracking mating partners. They observed the evolution of polymorphic evolutionary stable strategies. Finally, it is interesting to briefly mention the field of environment-driven evolution, which we briefly mentioned in the Introduction. This branch of ER differs from the general field by its more "realistic" approach to evolution. This paradigm is quite new~\parencite{Bredeche2010, Bredeche2012} but some have already been interested in the benefits of this paradigm by studying the evolution of altruism and dispersal strategies~\parencite{Montanier2011, Montanier2013}.

    Again, as cooperation is one of the major puzzle in the evolution of social behaviours, a part of the studies modeling evolution in ER has been interested more directly on the emergence of cooperation. In particular, as it is classical when studying cooperation, most of the work has been focused on the evolution of altruism. Waibel \& colleagues~\parencite{Waibel2011} conducted an "empirical" test of Hamilton's rule\footnote{As a quick reminder, Hamilton's rule states that altruism can evolve if the following inequation is respected: $rb > c$, where $b$ and $c$ are respectively the fitness benefits on the recipient and costs on the actor of the cooperative interaction and $r$ is the relatedness between the recipient and the actor.}~\parencite{Hamilton1964} for the evolution of altruism in a group of robots. In particular, they use a foraging task to quantitatively alter the costs and benefits of the cooperation action. They then test the influence of the coefficient of relatedness between robots on the evolution of altruism. They show that, in this context, Hamilton's rule is indeed quantitatively validated. Similarly in another study they have been interested on the influence of the genetic composition of groups of robots and the level of selection in the evolution of cooperation~\parencite{Waibel2009}. In particular, they studied the differences between groups of related or unrelated individuals and between selection acting at the level of the individual or that of the group. As could be expected, teams where individuals were strongly related performed better in tasks which required cooperation. Montanier and Bredeche have also studied the evolution of altruism but in an environment-driven context (as presented before). In particular, in one study~\parencite{Montanier2011}, they have studied the evolution of altruism in a situation of tragedy of commons~\parencite{Hardin1968}. This means that individuals have to share a common limited ressources to the point that some individuals have to sacrifice themselves for the population to survive. They showed that altruism can evolve under sufficient genetic relatedness. In another similar study~\parencite{Montanier2013}, the same authors validate the existence of a negative correlation between the evolution of altruism and spatial dispersion. To put it more simply, under low dispersion, individuals will tend to interact with other individuals nearby. Thus they interact with individuals that are more genetically related to them, which enables kin selection to take place~\parencite{VanBaalen1998}. Others have been interested in the evolution of division of labour (or specialization), where individuals are able to divide between different roles. These studies have in particular been interested in the evolution of specialization in ants. For example, Ferrante et al.~\parencite{Ferrante2015} have developed an evolutionary robotics model of task-partioning (where a task has to be done in sequence) in leafcutter ants. In this species, some ants are tasked with cutting leaves and leave them in a storage location from which other ants collect the leaves and bring them back to the nest. They showed that division of labour could evolve when particular environmental features (a slope in their case) could be exploited to reduce switching costs. They thus validated the theory about the role of switching costs in the evolution of specialization~\parencite{Duarte2011}. Finally, it is interesting to mention again the work of Solomon et al.~\parencite{Solomon2012} as their model on hyenas' signaling deals with the evolution of cooperation. More importantly, while they focus on the evolution of signaling strategies, these strategies are used to evolve efficient coordination. They also use a group of unrelated individuals, which is rare in ER. They are among the few to have been interested in the evolution of coordination among unrelated individuals. 


  \subsection{Conclusions}

    One thing that should now be clear is that we do not claim that any model to be fundamentally better than the others. In the contrary, and even if there has been a progression in the way we deal with evolutionary processes, each model is useful in its own way to study a particular problem given a set of assumptions and a level of abstraction. Even considering a particular model to be globally more realistic can be tricky; some models simply represent more accurately a defined aspect of the phenomenon. Mitri and colleagues~\parencite{Mitri2012} chose to classify different models for the study of social behaviours according to their situatedness, which they defined as "the extent to which individuals are embedded in an environment that they can sense and modify" (see Figure TODO).

    In the case of the problem posed in this chapter, we need a framework which could model the mechanistic constraints in the evolution of coordinative behaviours. In particular, the basis of our study is the model of the stag hunt. This could beg the question of why not simply use EGT. The answer is that the models in EGT tend to make critical assumptions on how strategies evolve. In particular, in the case of the evolution of cooperation, the problem of evolving coordination is only rarely mentioned~\parencite{Forber2015}. We just assume that both coordination and cooperation evolve together. Yet, as we previously talked about in the Introduction, without the simultaneous evolution of cooperation of all individuals, no benefits can be reaped from a collective actions. Moreover, evolving coordination implies the emergence of a complex behaviour where the individuals are capable to react to each other in order to act accordingly. This means that, in comparison to what is often assumed in EGT (and adaptive dynamics), the mutations necessary to switch from a solitary behaviour to a cooperative one are not easily available. This can have substantial effects on the ultimate evolution of cooperation. Studying these effects require a model which can study the mechanistic constraints of these coordinative behaviours: this is the only way to truly consider the proximate causes of cooperation in the stag hunt.

    In consequence, we need an individual-based study of adaptive behaviours; this means that a classical IBM could in theory be sufficient to that end. However, we find the point made by Mitri et al.~\parencite{Mitri2012} with relation to the advantages of ER in comparison to IBM to be quite accurate. In particular we are interested in the mechanics of coordinative behaviour. But the manner in which individuals coordinate can be really diverse. Observing these variations in coordinative strategies can be of great interest in order to study the effect of behaviours on the evolution of cooperation. When the individuals are embodied as they are in ER, the range of their possibilities in the way to coordinate is wider. Curiously, this also means that the coordinative behaviours may be simpler and may not require any particular sensory capabilities or the capacity to communicate. In any way, we do not want to restrict the diversity of behaviours available. Finally, it is important to make clear that we do not use any physical robots. Real robots would give the advantage of ensuring that any result is not imputable to artefacts of the simulation. This would also easily provide real physics (like friction and physical interactions) to the study; we however do not believe that to be of utmost importance in our case. At the moment, experiments on physical robots are unfortunately too time-consuming to consider using them to such scope~\parencite{Mitri2012, Doncieux2015}.



    % Là faudrait caser un jour une réflexion sur l'abstraction de notre modèle et l'approche animat. En gros répondre à Webb.
